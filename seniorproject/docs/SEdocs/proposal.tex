\documentclass{article}

\usepackage{fancyhdr}
\setlength{\headheight}{12pt}
\setlength{\textwidth}{17.2cm} \setlength{\textheight}{23cm}
\setlength{\topmargin}{-2.5cm} \setlength{\headsep}{1.6cm}
\setlength{\evensidemargin}{-.8cm}
\setlength{\oddsidemargin}{-.8cm}
%\pagestyle{fancy}

%set-up page dimentions
\usepackage[top=1.5 in, bottom = 1.5 in ,left = 1.5 in, right = 1.5in]{geometry}

\setlength{\parskip}{12pt}  % 12 pt = space between paragraphs
\setlength{\parindent}{0pt} % 0 pt  = indentation
\usepackage{amsmath}
\usepackage{amssymb}
\usepackage{amsthm}
\usepackage{ifthen}
\usepackage{latexsym}
\usepackage{graphicx}
\usepackage{graphics}
\usepackage{psfrag}
\usepackage{graphpap}
\usepackage{setspace}
\renewcommand{\P}{\text{P}}
\newcommand{\C}{\text{C}}


%\newcommand{\natnums}{{\mathbb N}}
%\newcommand{\algnums}{{\mathbb A}}
%\newcommand{\rationals}{{\mathbb Q}}
%\newcommand{\reals}{{\mathbb R}}
%\newcommand{\norm}[1]{\left|\left|#1\right|\right|}
%\newcommand{\unorm}[1]{{\left|\left|#1\right|\right|_u}}
%\newcommand{\scriptR}{\mathcal{R}}
%\newcommand{\scriptP}{\mathcal{P}}
%\newcommand{\taggedP}{\dot{\mathcal{P}}}
%\newcommand{\scriptQ}{\mathcal{Q}}
%\newcommand{\taggedQ}{\dot{\mathcal{Q}}}


% Allows hyperlinks if compiled with pdflatex
\usepackage{hyperref}
\hypersetup{colorlinks}
\usepackage{color}
\definecolor{darkred}{rgb}{0.5,0,0}
\definecolor{darkgreen}{rgb}{0,0.5,0}
\definecolor{darkblue}{rgb}{0,0,0.5}
\hypersetup{ colorlinks,
                linkcolor=darkblue,
                filecolor=darkgreen,
                urlcolor=darkblue,
                citecolor=darkblue }
%hyperlink example is: \href{http://www.google.com}{google}

%add code!
\usepackage{listings}
\definecolor{mygreen}{rgb}{0,0.6,0}
\definecolor{mygray}{rgb}{0.5,0.5,0.5}
\definecolor{mymauve}{rgb}{0.58,0,0.82}
\lstset{ %
backgroundcolor=\color{white},   % choose the background color; you must add \usepackage{color} or \usepackage{xcolor}
basicstyle=\footnotesize,        % the size of the fonts that are used for the code
breakatwhitespace=false,         % sets if automatic breaks should only happen at whitespace
breaklines=true,                 % sets automatic line breaking
captionpos=b,                    % sets the caption-position to bottom
commentstyle=\color{mygreen},    % comment style
deletekeywords={...},            % if you want to delete keywords from the given language
escapeinside={\%*}{*)},          % if you want to add LaTeX within your code
extendedchars=true,              % lets you use non-ASCII characters; for 8-bits encodings only, does not work with UTF-8
frame=single,                    % adds a frame around the code
keepspaces=true,                 % keeps spaces in text, useful for keeping indentation of code (possibly needs columns=flexible)
keywordstyle=\color{blue},       % keyword style
language=C,                 % the language of the code
morekeywords={*,...},            % if you want to add more keywords to the set
numbers=left,                    % where to put the line-numbers; possible values are (none, left, right)
numbersep=5pt,                   % how far the line-numbers are from the code
numberstyle=\tiny\color{mygray}, % the style that is used for the line-numbers
rulecolor=\color{black},         % if not set, the frame-color may be changed on line-breaks within not-black text (e.g. comments (green here))
showspaces=false,                % show spaces everywhere adding particular underscores; it overrides 'showstringspaces'
showstringspaces=false,          % underline spaces within strings only
showtabs=false,                  % show tabs within strings adding particular underscores
stepnumber=2,                    % the step between two line-numbers. If it's 1, each line will be numbered
stringstyle=\color{mymauve},     % string literal style
tabsize=2,                       % sets default tabsize to 2 spaces
title=\lstname                   % show the filename of files included with \lstinputlisting; also try caption instead of title
}


\begin{document}

\begin{center}
\Huge
Project Proposal\\ \vspace{2cm}
\Large
Nicholas Otto\\
Advisor: Dr. John Glick\\
\today
\end{center}

\newpage
\section{General Description}
For my senior project I will be implementing a C compiler called "NOCC" targeting the ARMv6 architecture.  
NOCC includes an implementation of the C preprocessor, a scanner, parser, intermediate code representation, code generator, and assembler.  
The goal of this project is to be able to completely describe a sufficient process which converts an arbitrary C program into machine executable code.

\section{Timeline and Deliverables}
The following timeline is subject to significant changes, but gives some rough targets.  
All documents are rough drafts until final submission.

\begin{tabular}{l l}
Requirements and Specification & 2/13\\
Design Document & 2/13\\
Test-plan & 2/20\\
Completed implementation & 5/1\\
Test-report & 5/8
\end{tabular}

\section{Project Plan}
Significant time will be spent researching standard compiler design techniques.  
I plan to make NOCC as modular as possible in order to have something completed in case the full compiler is not completed.
The modules comprising NOCC (subject to change) will be the preprocessor, scanner and parser, code generator, and assembler.
Well-defined interfaces should allow each module to be run independently of one another.

Technically, each of the modules could be considered a separate project.
The preprocessor I currently envision is mostly a "find and replace" program which finds preprocessor statements like \texttt{\#define} and then makes the appropriate substitutions.
The scanner and parser - the "front-end" - will be created as a module that takes in a "properly formatted C file" (that is a C file that would be unchanged if ran through the preprocessor module) and output some kind of description of the program to the back-end module (the exact format of this output is to be determined).
The code generator back-end will be implemented as another separate module that takes in input from the front-end and can output an assembly source file which is the compiled original C program.
The assembler, similar to the preprocessor, is a much more direct conversion of the input file (assembly source) to object code.

NOCC itself will be written in C and is intended to work on a "minimal operating system."
Thus, NOCC will be written using as few library functions as possible. 
Every attempt will be made to only include the standard input/output library functions as well as \texttt{malloc}, \texttt{free}, and \texttt{exit} from the standard library.

Time permitting, every module will be completed.
However, I recognize that there is much more to a compiler than just the program itself (namely operating system support).
Thus, one strategy I envision for testing the correctness of the compiler is to test the assembly source code file produced by my program, but then assemble that file with a standard assembler (probably of the gcc variant).
This object code (which since assembled by a more robust assembler) should be able to more easily demonstrate the correctness of the compiler.

In addition to Dr. Glick's help and various on-line resources, \textit{The C Programming Language}\cite{kr}, the "Red Dragon Book\cite{dragon}," and \textit{Engineering a Compiler}\cite{eac} will be used as references.


\bibliography{references}
\bibliographystyle{plain}
\end{document}

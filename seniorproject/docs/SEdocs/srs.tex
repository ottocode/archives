\documentclass{article}

\usepackage{fancyhdr}
\setlength{\headheight}{12pt}
\setlength{\textwidth}{17.2cm} \setlength{\textheight}{23cm}
\setlength{\topmargin}{-2.5cm} \setlength{\headsep}{1.6cm}
\setlength{\evensidemargin}{-.8cm}
\setlength{\oddsidemargin}{-.8cm}
%\pagestyle{fancy}

%set-up page dimentions
\usepackage[top=1.5 in, bottom = 1.5 in ,left = 1.5 in, right = 1.5in]{geometry}

\setlength{\parskip}{12pt}  % 12 pt = space between paragraphs
\setlength{\parindent}{0pt} % 0 pt  = indentation
\usepackage{amsmath}
\usepackage{amssymb}
\usepackage{amsthm}
\usepackage{ifthen}
\usepackage{latexsym}
\usepackage{graphicx}
\usepackage{graphics}
\usepackage{psfrag}
\usepackage{graphpap}
\usepackage{setspace}

% Allows hyperlinks if compiled with pdflatex
\usepackage{hyperref}
\hypersetup{colorlinks}
\usepackage{color}
\definecolor{darkred}{rgb}{0.5,0,0}
\definecolor{darkgreen}{rgb}{0,0.5,0}
\definecolor{darkblue}{rgb}{0,0,0.5}
\hypersetup{ colorlinks,
                linkcolor=darkblue,
                filecolor=darkgreen,
                urlcolor=darkblue,
                citecolor=darkblue }
%hyperlink example is: \href{http://www.google.com}{google}

%add code!
\usepackage{listings}
\definecolor{mygreen}{rgb}{0,0.6,0}
\definecolor{mygray}{rgb}{0.5,0.5,0.5}
\definecolor{mymauve}{rgb}{0.58,0,0.82}
\lstset{ %
backgroundcolor=\color{white},   % choose the background color; you must add \usepackage{color} or \usepackage{xcolor}
basicstyle=\footnotesize,        % the size of the fonts that are used for the code
breakatwhitespace=false,         % sets if automatic breaks should only happen at whitespace
breaklines=true,                 % sets automatic line breaking
captionpos=b,                    % sets the caption-position to bottom
commentstyle=\color{mygreen},    % comment style
deletekeywords={...},            % if you want to delete keywords from the given language
escapeinside={\%*}{*)},          % if you want to add LaTeX within your code
extendedchars=true,              % lets you use non-ASCII characters; for 8-bits encodings only, does not work with UTF-8
frame=single,                    % adds a frame around the code
keepspaces=true,                 % keeps spaces in text, useful for keeping indentation of code (possibly needs columns=flexible)
keywordstyle=\color{blue},       % keyword style
language=C,                 % the language of the code
morekeywords={*,...},            % if you want to add more keywords to the set
numbers=left,                    % where to put the line-numbers; possible values are (none, left, right)
numbersep=5pt,                   % how far the line-numbers are from the code
numberstyle=\tiny\color{mygray}, % the style that is used for the line-numbers
rulecolor=\color{black},         % if not set, the frame-color may be changed on line-breaks within not-black text (e.g. comments (green here))
showspaces=false,                % show spaces everywhere adding particular underscores; it overrides 'showstringspaces'
showstringspaces=false,          % underline spaces within strings only
showtabs=false,                  % show tabs within strings adding particular underscores
stepnumber=2,                    % the step between two line-numbers. If it's 1, each line will be numbered
stringstyle=\color{mymauve},     % string literal style
tabsize=2,                       % sets default tabsize to 2 spaces
title=\lstname                   % show the filename of files included with \lstinputlisting; also try caption instead of title
}

\newcommand{\NOCCVersion}{0.0.1}

\begin{document}
\begin{center}
\Huge
Software Requirements Specification\\\vspace{2cm}
\texttt{NOCC}\\\vspace{1cm}
\Large
Version \NOCCVersion\\
\today
\end{center}

\normalsize
\newpage
\section{Introduction}
\subsection{Purpose}
NOCC is a C compiler to be written by Nicholas Otto.

NOCC is sometimes an acronym for "non-optimized C compiler."  
As the name describes, NOCC is simply an attempt to create a working C compiler.  
All work until the version 1 release of NOCC (senior project completion) will be targeting the ARMv6 architecture.  
The goal of NOCC is simply to prove that a working C compiler is within the scope of an undergraduate-level project.

\subsection{References}
Kernighan and Ritchie's \textit{The C Programming Language, 2\textsuperscript{nd} Edition} is the C reference being used to create NOCC.
Also consulted are Aho, Sethi, and Ullman's \textit{Compilers.  Principles, Techniques, and Tools} as well as Cooper and Torczon's \textit{Engineering a Compiler}.

\section{Overall Description}
From the user's perspective, NOCC is a very simple program.  
Given a file of C source code, NOCC will produce ARM assembly output which represents the original source code in the assembly language form.
Users will have the option to produce an intermediate file after the preprocessing stage.

\section{External Interface Requirements}
In order to function properly, NOCC assumes the host operating system has a standard C implementation available.  
However, only the \texttt{stdio} and \texttt{stdlib} header files will be used by NOCC.

\section{System Features}
Currently, only certain preprocessor-related features are envisioned.  
Otherwise, NOCC will run to completion after compilation begins without further interaction or customization from the user.

\subsection{Preprocessor Options}
The NOCC preprocessor executes before any of the other compilation steps.  
Preprocessing involves scanning the source code and interpreting any of the preprocessing directives (such as \texttt{\#include, \#define}, etc).
The normal output of the preprocessor is simply a temporary file pointer to a file containing the original source code with the preprocessor directives carried out.
The user can prevent NOCC from deleting the temporary file with a command line option when running NOCC.
Additionally, the user can have NOCC stop compilation after the preprocessing stage (effectively making NOCC a preprocessor only).
This option will always produce the preprocessing output, even if the user did not specify to do so explicitly.

\section{Nonfunctional Requirements}
NOCC is a fairly simple, non-optimized C compiler.  
Therefore the execution time should be on par with that of larger commercial efforts (like gcc or llvm).
However, no guarantee is made about the execution time of the compiled executable.

Similarly, the size of the output file produced by NOCC should not be unreasonably different than that of gcc.
If the output file is far larger than the gcc version, some effort will be spent investigating the cause.

\section{Glossary}
\begin{tabular}{l p{12cm}}
NOCC & Non-Optimized C Compiler \\
ARM & Formally "Advanced RISC Machines."  Create processor designs and interface specifications that are popular with mobile/embedded devices \\
gcc & GNU Compiler Collection.  Contains an industry standard, open-source C compiler.\\
llvm & A project which created a compiler originally designed at the University of Illinois.  Often associated with the Clang compiler.
\end{tabular}

\end{document}

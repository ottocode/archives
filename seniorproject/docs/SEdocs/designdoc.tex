\documentclass{article}

\usepackage{fancyhdr}
\setlength{\headheight}{12pt}
\setlength{\textwidth}{17.2cm} \setlength{\textheight}{23cm}
\setlength{\topmargin}{-2.5cm} \setlength{\headsep}{1.6cm}
\setlength{\evensidemargin}{-.8cm}
\setlength{\oddsidemargin}{-.8cm}
%\pagestyle{fancy}

%set-up page dimentions
\usepackage[top=1.5 in, bottom = 1.5 in ,left = 1.5 in, right = 1.5in]{geometry}

\setlength{\parskip}{12pt}  % 12 pt = space between paragraphs
\setlength{\parindent}{0pt} % 0 pt  = indentation
\usepackage{amsmath}
\usepackage{amssymb}
\usepackage{amsthm}
\usepackage{ifthen}
\usepackage{latexsym}
\usepackage{graphicx}
\usepackage{graphics}
\usepackage{psfrag}
\usepackage{graphpap}
\usepackage{setspace}
\renewcommand{\P}{\text{P}}
\newcommand{\C}{\text{C}}


\newcommand{\natnums}{{\mathbb N}}
\newcommand{\algnums}{{\mathbb A}}
\newcommand{\rationals}{{\mathbb Q}}
\newcommand{\reals}{{\mathbb R}}
\newcommand{\norm}[1]{\left|\left|#1\right|\right|}
\newcommand{\unorm}[1]{{\left|\left|#1\right|\right|_u}}
\newcommand{\scriptR}{\mathcal{R}}
\newcommand{\scriptP}{\mathcal{P}}
\newcommand{\taggedP}{\dot{\mathcal{P}}}
\newcommand{\scriptQ}{\mathcal{Q}}
\newcommand{\taggedQ}{\dot{\mathcal{Q}}}


% Allows hyperlinks if compiled with pdflatex
\usepackage{hyperref}
\hypersetup{colorlinks}
\usepackage{color}
\definecolor{darkred}{rgb}{0.5,0,0}
\definecolor{darkgreen}{rgb}{0,0.5,0}
\definecolor{darkblue}{rgb}{0,0,0.5}
\hypersetup{ colorlinks,
                linkcolor=darkblue,
                filecolor=darkgreen,
                urlcolor=darkblue,
                citecolor=darkblue }
%hyperlink example is: \href{http://www.google.com}{google}

%add code!
\usepackage{listings}
\definecolor{mygreen}{rgb}{0,0.6,0}
\definecolor{mygray}{rgb}{0.5,0.5,0.5}
\definecolor{mymauve}{rgb}{0.58,0,0.82}
\lstset{ %
backgroundcolor=\color{white},   % choose the background color; you must add \usepackage{color} or \usepackage{xcolor}
basicstyle=\footnotesize,        % the size of the fonts that are used for the code
breakatwhitespace=false,         % sets if automatic breaks should only happen at whitespace
breaklines=true,                 % sets automatic line breaking
captionpos=b,                    % sets the caption-position to bottom
commentstyle=\color{mygreen},    % comment style
deletekeywords={...},            % if you want to delete keywords from the given language
escapeinside={\%*}{*)},          % if you want to add LaTeX within your code
extendedchars=true,              % lets you use non-ASCII characters; for 8-bits encodings only, does not work with UTF-8
frame=single,                    % adds a frame around the code
keepspaces=true,                 % keeps spaces in text, useful for keeping indentation of code (possibly needs columns=flexible)
keywordstyle=\color{blue},       % keyword style
language=C,                 % the language of the code
morekeywords={*,...},            % if you want to add more keywords to the set
numbers=left,                    % where to put the line-numbers; possible values are (none, left, right)
numbersep=5pt,                   % how far the line-numbers are from the code
numberstyle=\tiny\color{mygray}, % the style that is used for the line-numbers
rulecolor=\color{black},         % if not set, the frame-color may be changed on line-breaks within not-black text (e.g. comments (green here))
showspaces=false,                % show spaces everywhere adding particular underscores; it overrides 'showstringspaces'
showstringspaces=false,          % underline spaces within strings only
showtabs=false,                  % show tabs within strings adding particular underscores
stepnumber=2,                    % the step between two line-numbers. If it's 1, each line will be numbered
stringstyle=\color{mymauve},     % string literal style
tabsize=2,                       % sets default tabsize to 2 spaces
title=\lstname                   % show the filename of files included with \lstinputlisting; also try caption instead of title
}

\newcommand{\NOCCVersion}{0.0.1}

\begin{document}
\begin{center}
\Huge
Design Document\\\vspace{2cm}
\texttt{NOCC}\\\vspace{1cm}
\Large
Version \NOCCVersion\\
\today
\end{center}

\newpage
\section{Introduction}
To simplify design, NOCC will be designed with the singular focus of creating a working compiler that targets the ARMv6 architecture.
Whereas other compilers (namely LLVM) strive to be highly modular and allow for swapping out "font" and "back-ends," any such flexibility found in NOCC is purely accidental.
The most limited resource in developing NOCC is time, a fact which will drive all development and design decisions.

NOCC essentially runs two separate programs, the preprocessor and the rest of the compiler.
The output of the preprocessor is only another source code file - but with all preprocessor directives carried out.
No data-structures or other variables will be carried over from the preprocessor stage to the remaining execution.

\section{Data Design}

\subsection{Preprocessor}
The preprocessor symbol table will be pre-seeded with the following keywords:
\begin{center}
\begin{tabular}{c c c}
\#  &   define & error\\
include &   ifdef & pragma\\
ifndef &   undef & line\\
elif    &   else & endif\\
\end{tabular}
\end{center}

\subsection{NOCC}
Global structures for NOCC include the symbol table, the parse tree, and others to be determined.

The scanner will de-construct the source code into a stream of tokens.
Tokens are C structures that contain the following information:
\begin{itemize}
\item type. The token class (keyword, identifier, constant, string literal, operator, or other).
\item file\_num. The file this token was taken from (for cases when multiple files are being compiled).
\item line\_num.  The line which the token occurred.
\item lexeme\_length.  The number of characters which make up the lexeme.
\item lexeme.  The actual token characters.
\item More to be determined.
\end{itemize}

The symbol table for NOCC will be implemented as a hash-table pre-seeded with the following keyword tokens:
\begin{center}
\begin{tabular}{c c c}
auto &break & case\\
char & const & continue\\
default & do & double\\
else & enum & extern\\
float & for & goto\\
if & int & long\\
register & return & short\\
signed & sizeof & static\\
struct & switch & typedef\\
union & unsigned & void\\
volatile & while  & \\
\end{tabular}
\end{center}

\section{Architectural and Component-Level Design}
While large portions of the architecture still need to be worked out, NOCC can be split into the following groups:
\begin{itemize}
\item the preprocessor
\item the scanner
\item the parser/code-generator
\item and the back-end.
\end{itemize}
A small amount of control code will exist on top of the  groups to handle proper program control flow (for example, calling the parser/code-generator when preprocessing is finished).
See attached appendix for a rough drawing of the current architecture set-up.

\subsection{Preprocessor Interface Specification}
The preprocessor can be started using the following interface:\\ \\ 
\texttt{FILE * begin\_preprocessing(char *file\_name, char args);}\\ \\
where \texttt{file\_name} contains the absolute path to the file to pre-process and \texttt{args} is greater than 0 to produce preprocessor output.
The returned file pointer will be the output of the preprocessor.

\subsection{Scanner Interface Specification}
The scanner is implemented as a separate group because it will be called directly by both the preprocessor and parser/code-generator.
The interface is given by the following header file:

\begin{lstlisting}
#include <stdlib.h>
#include <stdio.h>
#include "../utility/structures.h"

enum  STATES{EMPTY, ID_KW, 
            C_NUM, C_NUMU, C_NUML, C_NUMUL, 
            C_HEX0, C_HEX1, 
            C_DF, C_FE, C_FS, C_FED, C_FF, C_FL, C_F,
            C_CO, C_CE, C_CD, C_CC,
            S_O, S_E, S_D, S_C,
            OP, OP1, OP2};

/** Convert Enum To String
 *  -----------
 *  For debuggin purposes.
 *
 *  state: the state of the scanner
 *
 *  returns: "state" 
 */
const char* getEnumName(enum STATES state);


/** Open a File
 *  -----------
 *  Opens a file and assigns pointer to file_p.
 *      If file_p points to a value other than NULL, a 2 is returned.
 *      Thus, it is up to the caller to ensure that all previously opened
 *      files have been closed
 *
 *  filename:  the name of file to open
 *  args:  the file attributes with which to open
 *
 *  returns:  0 if file successfully opened, 1 otherwise.
 */
int open_file(const char *filename, const char *args);

/** Close a File
 *  ------------
 *  Closes the file pointed to by file_p.
 *      Assignes file_p to NULL.
 *      It is not an error to call close_file if file_p is already NULL
 *
 *  returns:  0 on success, 1 on error
 */
int close_file();

/** Get Remaining Characters
 *  ------------------------
 *  Determines the number of characters remaining after lb (lexeme beginning)
 *
 *  returns:  number of characters remaining.
 *              If no file is currently opened, then 0 is returned.
 */
unsigned long remaining_characters();

/** Get Next Token
 *  --------------
 *  Get the next token in the file
 *
 *  returns:  a pointer to a token structure corresponding to next character sequence in file.
 *              If no file is currently opened, then NULL is returned.
 *              If token cannot be parsed, then (void *)1 is returned.
 */
struct token *get_next_token();

#endif
\end{lstlisting}

\subsection{Other Interface Specifications}
The interface for the other groups is still to be determined.  It is entirely possible that more groups will be created as the required task becomes clearer.


\section{User Interface Design}
The user interface for NOCC is a Unix style command line.  The standard interaction is given by:\\ \\
\texttt{nocc file.c}\\ \\
where \texttt{file.c} is the file to be compiled.
The optional preprocessor commands, halt after preprocessing and produce preprocessor output will be given by:\\ \\
\texttt{nocc -po file.c}\\ \\
and\\ \\
\texttt{nocc -ph file.c}\\ \\
respectively.

\section{Testing}
Testing is to be a continuous process along with development.
As new features and capabilities are added, tests will be immediately (if not concurrently) written to ensure proper execution and functionality.
While testing will not be robust enough to provably show correctness of the compiler, it should give reasonable assurances that the compiled output is what the original programmer intended.
Additionally, testing should be sufficient to indicate areas where NOCC may perform incorrectly.

\end{document}

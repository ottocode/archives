\documentclass{article}

\usepackage{fancyhdr}
\setlength{\headheight}{12pt}
\setlength{\textwidth}{17.2cm} \setlength{\textheight}{23cm}
\setlength{\topmargin}{-2.5cm} \setlength{\headsep}{1.6cm}
\setlength{\evensidemargin}{-.8cm}
\setlength{\oddsidemargin}{-.8cm}
%\pagestyle{fancy}

%set-up page dimentions
\usepackage[top=1.5 in, bottom = 1.5 in ,left = 1.5 in, right = 1.5in]{geometry}

\setlength{\parskip}{12pt}  % 12 pt = space between paragraphs
\setlength{\parindent}{0pt} % 0 pt  = indentation
\usepackage{amsmath}
\usepackage{amssymb}
\usepackage{amsthm}
\usepackage{ifthen}
\usepackage{latexsym}
\usepackage{graphicx}
\usepackage{graphics}
\usepackage{psfrag}
\usepackage{graphpap}
\usepackage{setspace}
\renewcommand{\P}{\text{P}}
\newcommand{\C}{\text{C}}


\newcommand{\nonterm}[1]{{\textit{#1:}\\ }}
\newcommand{\production}[1]{{\hspace*{1cm}#1\\ }}

\newcommand{\psymbol}[1]{{\textit{#1}}}
\newcommand{\terminal}[1]{{\texttt{#1}}}
\newcommand{\optional}[1]{{#1\(_\text{opt}\) }}

\newcommand{\natnums}{{\mathbb N}}
\newcommand{\algnums}{{\mathbb A}}
\newcommand{\rationals}{{\mathbb Q}}
\newcommand{\reals}{{\mathbb R}}
\newcommand{\norm}[1]{\left|\left|#1\right|\right|}
\newcommand{\unorm}[1]{{\left|\left|#1\right|\right|_u}}
\newcommand{\scriptR}{\mathcal{R}}
\newcommand{\scriptP}{\mathcal{P}}
\newcommand{\taggedP}{\dot{\mathcal{P}}}
\newcommand{\scriptQ}{\mathcal{Q}}
\newcommand{\taggedQ}{\dot{\mathcal{Q}}}


% Allows hyperlinks if compiled with pdflatex
\usepackage{hyperref}
\hypersetup{colorlinks}
\usepackage{color}
\definecolor{darkred}{rgb}{0.5,0,0}
\definecolor{darkgreen}{rgb}{0,0.5,0}
\definecolor{darkblue}{rgb}{0,0,0.5}
\hypersetup{ colorlinks,
                linkcolor=darkblue,
                filecolor=darkgreen,
                urlcolor=darkblue,
                citecolor=darkblue }
%hyperlink example is: \href{http://www.google.com}{google}

%add code!
\usepackage{listings}
%\definecolor{mygreen}{rgb}{0,0.6,0}
%\definecolor{mygray}{rgb}{0.5,0.5,0.5}
%\definecolor{mymauve}{rgb}{0.58,0,0.82}

\begin{document}

Senior project writeup

\section{NIC Programming Language}
NIC stands for "Nick's idealized C."  
The language is idealized in the sense that is is much easier to write a compiler for NIC than C - not that NIC adds any improvements to the original C language

\subsection{NIC Grammar}
Ther grammar for NIC is based off the C90 grammar specified in \cite{cpl}.
Numerous modifications were made to make it easier to construct an LL compiler by hand.
Symbols in \texttt{typewriter font} are considered terminal symbols or placeholders for a terminal token (like \texttt{identifier} which represents a token of type identifier - not literally "identifier").

\nonterm{compound-statement}
\production{\terminal{\{} \optional{declaration-list} \optional{statement-list} \terminal{\}}}
\nonterm{declaration-list}
\production{declaration}
\production{delclaration declaration-list}
\nonterm{declaration}
\production{declaration-specifiers typedef-name \terminal{;}}
\nonterm{declaration-specifiers}
\production{\optional{type-qualifier} type-specifier \optional{pointer-qualifier}}
\nonterm{type-qualifier}
\production{\textit{one of} \terminal{const} \terminal{volatile} \terminal{signed} \terminal{unsigned}}
\nonterm{type-specifier}
\production{\textit{one of} \texttt{void char short int long float double}}
\nonterm{pointer-qualifier}
\production{\terminal{*}}
\production{\terminal{*} pointer-qualifier}
\nonterm{typedef-name}
\production{identifier}
\production{identifier \terminal{[} constant \terminal{]}}
\nonterm{statement-list}
\production{statement}
\production{statement statement-list}
\nonterm{statement}
\production{expression-statement}
\production{jump-statement}
\production{labeled-statement}
\nonterm{labeled-statement}
\production{\terminal{begin} identifier \terminal{:}}
\nonterm{jump-statement}
\production{\terminal{goto} identifier identifier \terminal{;}}
\nonterm{expression-statement}
\production{\optional{expression} \terminal{;}}
\nonterm{expression}
\production{assignment-expression}
\nonterm{assignment-expression}
\production{lvalue assignment-operator rvalue}
\nonterm{assignment-operator}
\production{\terminal{=}}
\nonterm{lrvalue}
\production{postfix-expresstion}
\production{reference-operator postfix-expression}
\nonterm{lvalue}
\production{lpostfix-expression}
\production{reference-operator lpostfix-expression}
\nonterm{rvalue}
\production{logical-OR-expression}
\production{logical-AND-expression}
\production{inclusive-OR-expression}
\production{exclusive-OR-expression}
\production{AND-expression}
\production{equality-expression}
\production{relational-expresion}
\production{shift-expression}
\production{additive-expression}
\production{multiplicative-expression}
\production{cast-expression}
\production{direct-expression}
\nonterm{reference-operator}
\production{\textit{one of} \terminal{\&} \terminal{*}}
\nonterm{postfix-expression}
\production{primary-expression}
\production{primary-expression \terminal{[} primary-expression \terminal{]}}
\production{primary-expression \terminal{(} \optional{primary-expression-list} \terminal{)}}
\production{primary-expression \terminal{.} identifier}
\nonterm{lpostfix-expression}
\production{identifier}
\production{identifier \terminal{[} primary-expression \terminal{]}}
\production{identifier \terminal{.} identifier}
\nonterm{primary-expression-list}
\production{primary-expression}
\production{primary-expression, primary-expression}
\nonterm{primary-expression}
\production{identifier}
\production{const}
\production{string}
\nonterm{logical-OR-expression}
\production{lrvalue \terminal{||} lrvalue}
\nonterm{logical-AND-expression}
\production{lrvalue \terminal{\&\&} lrvalue}
\nonterm{inclusive-OR-expression}
\production{lrvalue \terminal{|} lrvalue}
\nonterm{exclusive-OR-expression}
\production{lrvalue \terminal{\^} lrvalue}
\nonterm{AND-expression}
\production{lrvalue \terminal{\&} lrvalue}
\nonterm{equality-expression}
\production{lrvalue \terminal{==} lrvalue}
\production{lrvalue \terminal{!=} lrvalue}
\nonterm{relational-expresion}
\production{lrvalue \terminal{<} lrvalue}
\production{lrvalue \terminal{<=} lrvalue}
\production{lrvalue \terminal{>} lrvalue}
\production{lrvalue \terminal{>=} lrvalue}
\nonterm{shift-expression}
\production{lrvalue \terminal{>>} lrvalue}
\production{lrvalue \terminal{<<} lrvalue}
\nonterm{additive-expression}
\production{lrvalue \terminal{+} lrvalue}
\production{lrvalue \terminal{-} lrvalue}
\nonterm{multiplicative-expression}
\production{lrvalue \terminal{*} lrvalue}
\production{lrvalue \terminal{/} lrvalue}
\production{lrvalue \terminal{\%} lrvalue}
\nonterm{cast-expression}
\production{\terminal{(} type-specifier \terminal{)} lrvalue}
\nonterm{direct-expression}
\production{lrvalue}


\section{NOC - the NICC Compiler}
NOC is an acronym for "NIC's Original Compiler."
NOC is original in both the source code being custom built for this project, but also in the fact that it is the first compiler for the NIC language.

\section{\LaTeX tests}
Citation tests:\\
Kernighan and Richie is cited as \cite{cpl} 
The red dragon book is cited as \cite{reddragon}.

End citation test

Tests of commands:

A terminal symbol comes next \terminal{hereitis}.  

\bibliographystyle{plain}
\bibliography{sources}
\end{document}

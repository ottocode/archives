\documentclass{article}

\usepackage{fancyhdr}
\usepackage{enumitem} %allows enumerate resume
\setlength{\headheight}{12pt}
\pagestyle{fancy}

%set-up page dimentions
\usepackage[top=1.5 in, bottom = 1.5 in ,left = 1.5 in, right = 1.5in]{geometry}

\setlength{\parskip}{12pt} % 12 pt = space between paragraphs
\setlength{\parindent}{0pt} % 0 pt = indentation
\usepackage{amsmath}
\usepackage{amssymb}
\usepackage{amsthm}
\usepackage{ifthen}
\usepackage{latexsym}
\usepackage{graphicx}
\usepackage{graphics}
\usepackage{psfrag}
\usepackage{graphpap}
\usepackage{fullpage}
\renewcommand{\P}{\text{P}}
\newcommand{\C}{\text{C}}


% Allows hyperlinks if compiled with pdflatex
\usepackage{hyperref}
\hypersetup{colorlinks}
\usepackage{color}
\definecolor{darkred}{rgb}{0.5,0,0}
\definecolor{darkgreen}{rgb}{0,0.5,0}
\definecolor{darkblue}{rgb}{0,0,0.5}
\hypersetup{ colorlinks,
                linkcolor=darkblue,
                filecolor=darkgreen,
                urlcolor=darkblue,
                citecolor=darkblue }
%hyperlink example is: \href{http://www.google.com}{google}



\begin{document}
\begin{center}
ISA Design Concept Stage\\
Marc Slaughter and Nicholas Otto\\
February 11, 2013\\
\vspace{1.5em}
\hrule
\end{center}

\section{MIPS implementation of UltraMega}
\begin{verbatim}
# definitions:
# $t0 temp
# $a0 pc $a0 since first argument
# $a1 mem $a1 since second argument
# $t3 inst
# $t4 op
# $t5 srcA
# $t6 srcB
# $t7 dest

ultraMegaMips:
tagA:   ori $t0, $zero, 4
        mul $t0, $a0, $t0
        add $t3, $a1, $t0
        lw $t4, $t3 # Get $t4
        lw $t5, 4($t3) # Get $t5
        lw $t6, 8($t3) # Get $t6
        lw $t7, 12($t3) # Get $t7
        addi $a0, $a0, 4

########case 0######################
        bne $t4, $zero, tag1
        mul $t5, $t5, 4
        add $t0, $a1, $t5
        lw $t5, $t0 #$t5 = M[$t5]
        mul $t6, $t6, 4
        add $t0, $a1, $t6
        lw $t6, $t0 #$t6 = M[$t6]

        sub $t0, $t5, $t6
        mul $t7, $t7, 4
        add $t7, $a1, $t7
        sw $t0, $t7
        j tagA
        
        
########case 1######################
tag1:
        ori $t0, $zero, 1
        bne $t4, $t0, tag2
        mul $t5, $t5, 4
        add $t0, $a1, $t5
        lw $t5, $t0 #$t5 = M[$t5]
        srl $t5, $t5, 1

        mul $t7, $t7, 4 #get $t7
        add $t7, $a1, $t7
        sw $t5, $t7 #store
        j tagA
########case 2######################
tag2:
        ori $t0, $zero, 2
        bne $t4, $t0, tag3
        mul $t5, $t5, 4
        add $t0, $a1, $t5
        lw $t5, $t0 #$t5 = M[$t5]
        mul $t6, $t6, 4
        add $t0, $a1, $t6
        lw $t6, $t0 #$t6 = M[$t6]
        nor $t0, $t5, $t6
        
        mul $t7, $t7, 4 #get $t7
        add $t7, $a1, $t7
        sw $t0, $t7 #store
        j tagA
########case 3######################
tag3:
        ori $t0, $zero, 3
        bne $t4, $t0, tag4
        mul $t5, $t5, 4
        add $t0, $a1, $t5
        lw $t5, $t0 #$t5 = M[$t5]
        mul $t5, $t5, 4
        add $t0, $a1, $t5
        lw $t5, $t0 #$t5 = M[M[$t5]]
        mul $t6, $t6, 4
        add $t0, $a1, $t6
        lw $t6, $t0 #$t6 = M[$t6]
        mul $t7, $t7, 4 #get $t7
        add $t7, $a1, $t7
        sw $t7, $t5 # $t7 = MM$t5
        sw $t5, $t6 # MM$t5 = M$t6
        j tagA
########case 4######################
tag4:
        ori $t0, $zero, 4
        bne $t4, $t0, tag5
        mul $t5, $t5, 4
        add $t0, $a1, $t5
        lw $t5, $t0
        mul $t7, $t7, 4
        add $t0, $a1, $t7
        lw $t7, $t0 #$t7 = M[$t7]
        IN $t7, $t5
        j tagA
########case 5######################
tag5:
        ori $t0, $zero, 5
        bne $t4, $t0, tag6
        mul $t5, $t5, 4
        add $t0, $a1, $t5
        lw $t5, $t0
        mul $t6, $t6, 4
        add $t0, $a1, $t6
        lw $t6, $t0 #$t6 = M[$t6]
        OUT $t5, $t6
j tagA
########case 6######################
tag6:
        ori $t0, $zero, 6
        bne $t4, $t0, tag7
        mul $t5, $t5, 4
        add $t0, $a1, $t5
        lw $t5, $t0
        mul $t6, $t6, 4
        add $t0, $a1, $t6
        lw $t6, $t0
        mul $t7, $t7, 4
        add $t0, $a1, $t7
        lw $t7, $t0
        sw $t7, $a0
        bge $t5, $zero, tagA
        or $a0, $t6, 0
        j tagA
########case 7######################
tag7:
        ori $t0, $zero, 7
        bne $t4, $t0, tag8
        ori $v0, $a0, $zero # put $a0 i.e. pc into $v0 to return
        jr $ra
########no case#######################
tag8:
        j tagA
\end{verbatim}

\section{Our ISA implementation of UltraMega}

\begin{verbatim}
#define
pc = $1
mem = $2
inst = $3
op = $4
srcA = $5
srcB = $6
dest = $7
temp = $0
INT_SIZE = 1
#end


tagA: add inst, mem, pc
        lw op, inst // Get op
        lw srcA, 1(inst) // Get srcA
        lw srcB, 2(inst) // Get srcB
        lw dest, 3(inst) // Get dest
        set temp, 1
        add pc, pc, temp
       
********case 0**********************
        set temp, 0
        bne op, temp, tag1
        add srcA, mem, srcA
        lw srcA, srcA //srcA = M[srcA]
        add srcB, mem, srcB
        lw srcB, srcB //srcB = M[srcB]

        sub temp, srcA, srcB
        add dest, mem, dest
        sw temp, dest
        j tagA
 
        
********case 1**********************
tag1: set temp, 1
        bne op, temp, tag2
        add srcA, mem, srcA
        lw srcA, srcA //srcA = M[srcA]
        srl srcA, srcA, 1

        add dest, mem, dest
        sw srcA, dest //store
        j tagA

********case 2**********************
tag2: set temp, 2
        bne op, temp, tag3
        add srcA, mem, srcA
        lw srcA, srcA //srcA = M[srcA]
        add srcB, mem, srcB
        lw srcB, srcB //srcB = M[srcB]
        nor temp, srcA, srcB
        
        add dest, mem, dest
        sw temp, dest //store
        j tagA

********case 3**********************
tag3: set temp, 3
        bne op, temp, tag4
        add srcA, mem, srcA
        lw srcA, srcA //srcA = M[srcA]

        add temp, mem, srcA
        lw srcA, temp //srcA = M[M[srcA]]

        add srcB, mem, srcB
        lw srcB, srcB //srcB = M[srcB]
        add dest, mem, dest

        sw dest, srcA // dest = MMsrcA
        sw srcA, srcB // MMsrcA = MsrcB
        j tagA

********case 4**********************
tag4: set temp, 4
        bne op, temp, tag5
        add srcA, mem, srcA
        lw srcA, srcA
        add temp, mem, dest
        lw dest, temp //dest = M[dest]
        IN dest, srcA
        j tagA

********case 5**********************
tag5: set temp, 5
        bne op, temp, tag6
        add temp, mem, srcA
        lw srcA, temp

        mul srcB, srcB, INT_SIZE
        add temp, mem, srcB
        lw srcB, temp //srcB = M[srcB]
        OUT srcA, srcB
        j tagA

********case 6**********************
tag6: set temp, 6
       bne op, temp, tag7
        add temp, mem, srcA
        lw srcA, temp
        add temp, mem, srcB
        lw srcB, temp

        add temp, mem, dest
        lw dest, temp
        sw dest, pc
        set temp, 0
        slt srcA, srcA, temp
        beq srcA, temp, tagA

        set pc, srcB
        j tagA

********case 7**********************
tag7: QUIT pc
\end{verbatim}

\section{Fibonacci Implementation}

\begin{verbatim}

% Fibonacci implementation.  "n" is the top of the $sp

:fibo
    TRAP    6, $1       //put n in $1
    set     $2, 0        
    slt     $3, $1, $2  // $3 = if(n < 0)
    set     $2, 1
    beq     $2, $3, label1

    set     $2, 3
    slt     $3, $1, $2  // $3 = if(n < 3)
    set     $2, 1
    beq     $2, $3, label2
    //else if (n=29)
    set     $2, 29
    beq     $1, $2, label3
            
    set     $4, 1 //$3 is running total
    sub     $2, $1, $4  // $2 = n-1
    sub     $3, $2, $4  // $3 = n-2
    TRAP    5, $2       push n-1
    TRAP    5, $3       push n-2
    TRAP    3, fibo     call fibo
    TRAP    6, $1       $1 = pop
    TRAP    6, $2       $2 = pop
    TRAP    5, $1       push temp1
    TRAP    5, $2       push temp2
    TRAP    3, fibo     call fibo
    TRAP    6, $1       pop
    TRAP    6, $2       pop
    add     $0, $1, $2  add
    TRAP    5, $0       push
    TRAP    4, 0

label1:
    set     $1, 0
    set     $2, 1
    sub     $0, $1, $2  // -1   = b1111111111111111111
    TRAP    5, $0
    TRAP    4, 0

label2:  // 1    = b1
    set     $0, 1
    TRAP    5, $0
    TRAP    4, 0
    
label3:  // 514229 = b1111101100010110101
    set     $0, 0
    set     $1, 111110
    sl      $1, 13
    or      $0, $1, $0
    set     $1, 110001
    sl      $1, 7
    or      $0, $1, $0
    set     $1, 011010
    sl      $1, 1
    or      $0, $1, $0
    set     $1, 1
    add     $0, $0, $1
    TRAP    5, $0
    TRAP    4, 0

label4  // 832030 = b110010 110010 000111 10
    set     $0, 0
    set     $1, 110010
    sl      $1, 14
    or      $0, $1, $0
    set     $1, 110010
    sl      $1, 8
    or      $0, $1, $0
    set     $1, 000111
    sl      $1, 2
    or      $0, $1, $0
    set     $1, 2
    add     $0, $0, $1 
    TRAP    5, $0
    TRAP    4, 0


label5  // 4807526976 = b100011 110100 011010 000101 001000 000
    set     $0, 0
    set     $1, 100011
    sl      $1, 27
    or      $0, $1, $0
    set     $1, 110100
    sl      $1, 21
    or      $0, $1, $0
    set     $1, 011010
    sl      $1, 15
    or      $0, $1, $0
    set     $1, 000101
    sl      $1, 9
    or      $0, $1, $0
    set     $1, 001000
    sl      $1, 3
    or      $0, $1, $0
    TRAP    5, $0
    TRAP    4, 0



label6: ;7778742049 =  b111001 111101 001100 010111 100100 001
    set     $0, 0
    set     $1, 111001
    sl      $1, 27
    or      $0, $1, $0
    set     $1, 111101
    sl      $1, 21
    or      $0, $1, $0
    set     $1, 001100
    sl      $1, 15
    or      $0, $1, $0
    set     $1, 010111
    sl      $1, 9
    or      $0, $1, $0
    set     $1, 100100
    sl      $1, 3
    or      $0, $1, $0
    set     $1, 1
    add     $0, $1
    TRAP    5, $0
    TRAP    4, 0

\end{verbatim}



\section{Evaluation of EDP for MIPS and our ISA}
The general formula for the energy dely product is
\[EDP = d^2(5+W_I),\]
where \(d\) is the number if dynamic instructions and \(W_I\) is the number of bits per instructions.
For MIPS, we have 8 instructions to initialize the system, and for sake of convenience, let's say 10 instructions per execution.  
Thus, we have about 18 instructions per cycle, let's say 10 cycles, and an instruction length of 32 bits.  So
\[EDP_{MIPS} = 180^2(37) = 1198800.\]

To initialize the system, (up to case 0) we have 7 instructions.  Each case has between 8 and 14 instructions total (except for the exit case).  
Thus, each run through the system will have between 15 and 21 instructions.  For ease of use, let's just say 17 (the median value).  If we run our ISA 10 times, we have a total of 170 dynamic instructions for an EDP of:
\[EDP_{ours} = 170^2(21) = 606900.\]

\section{Analysis of Instructions}
Since there are only 16 instructions, the following is a description of the entire language (although some details are still to be fleshed out).\\

We have broken our ISA into two different operation types, similar to "R" and "I" type for MIPS. We have "Normal" and "S" type operations.
\subsection{Normal type operations}

Asembly format: \[opcode\; operand\; operand \;operand\]
where operand is a register, or an offset, depending on the opcode.
We have a 4bit op code and 3 bits for each operand
Thus, there is a possibility for 16 operations, and 8 general registers (the S type operations may access special registers or immediates, depending on the operation).

Operations:
\begin{enumerate}
\item
\begin{verbatim}
mul $dest, $op1, $op2
does what you think
returns lower 34 bits into $dest
    higher 34 bits goes to TEMP register
    $dest = lower34($op1 * $op2)
\end{verbatim}
\item
\begin{verbatim}
add $dest, $op1, $op2
    does what you think
    ignore overflow
    $dest = $op1 + $op2

\end{verbatim}
\item
\begin{verbatim}
sub $dest, $op1, $op2
    does what you think
    ignore overflow
    $dest = $op1 - $op2

\end{verbatim}
\item
\begin{verbatim}
or $dest, $op1, $op2
    logical or

\end{verbatim}
\item
\begin{verbatim}
nor $dest, $op1, $op2
~($op1 or $op2)

\end{verbatim}
\item
\begin{verbatim}
sr $dest, $op1, $op2
    bitwise shift-right
    $dest = $op1 >> $op2

\end{verbatim}
\item
\begin{verbatim}
lw $value, $base, $offset
    load word (34 bits).
    $value = M[$base + $offset]

\end{verbatim}
\item
\begin{verbatim}
sw $value, $base, $offset
    store word (34 bits)
    M[$base + $offset] = $value

\end{verbatim}
\item
\begin{verbatim}
bne $op1, $op2, PC_offset
branch not equal
if($op1 != $op2):
PC = PC+InstL+(InstL)*PC_offset
note: this will be handled by assembler so we can implement label functionality

\end{verbatim}
\item
\begin{verbatim}
beq $op1, $op2, PC_offset
branch if equal
if($op1 == $op2):
PC = PC+InstL+(InstL)*PC_offset
note: this will be handled by assembler so we can implement label functionality

\end{verbatim}
\item
\begin{verbatim}
slt $dest, $op1, $op2
    set if less than
    if($op1 < $op2):
        $dest = 1
else:
$dest = 0
\end{verbatim}
\end{enumerate}


\subsection{S TYPE}
Assembly format: \[opcode \;operand1 \;operand2\]
where opcode is a 4 bit op code, operand1 is 3 bit specifying normal register, and operand2 is a 6 bit special designator.
\$op will be used to specify a normal register, and \$s will specify a special register

\begin{enumerate}[resume]
\item
\begin{verbatim}
j $op1, $op2
    jump to PC + instruction length (13 bits) + $op1+$op2 instructions
    PC = PC+InstL+ InstL*($op1+$op2)

\end{verbatim}
\item
\begin{verbatim}
IN $dest, Ch#
store into $op from $s (where $s is the channel)

\end{verbatim}
\item
\begin{verbatim}
Out $source, Ch#
send $op to $s

\end{verbatim}
\item
\begin{verbatim}
QUIT $op1, 0
        return $op1 and exit program

\end{verbatim}
\item
\begin{verbatim}
set $dest, immediate
        $dest = 0 | immediate
\end{verbatim}
\end{enumerate}


\end{document}

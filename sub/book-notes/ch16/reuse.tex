\documentclass{article}

\usepackage{fancyhdr}
\setlength{\headheight}{12pt}
\setlength{\textwidth}{17.2cm} \setlength{\textheight}{23cm}
\setlength{\topmargin}{-2.5cm} \setlength{\headsep}{1.6cm}
\setlength{\evensidemargin}{-.8cm}
\setlength{\oddsidemargin}{-.8cm}
%\pagestyle{fancy}

%set-up page dimentions
\usepackage[top=1 in, bottom = 1 in ,left = 1 in, right = 1in]{geometry}

\setlength{\parskip}{12pt}  % 12 pt = space between paragraphs
\setlength{\parindent}{0pt} % 0 pt  = indentation
\usepackage{amsmath}
\usepackage{amssymb}
\usepackage{amsthm}
\usepackage{ifthen}
\usepackage{latexsym}
\usepackage{graphicx}
\usepackage{graphics}
\usepackage{psfrag}
\usepackage{graphpap}
\renewcommand{\P}{\text{P}}
\newcommand{\C}{\text{C}}


\newcommand{\natnums}{{\mathbb N}}
\newcommand{\algnums}{{\mathbb A}}
\newcommand{\rationals}{{\mathbb Q}}
\newcommand{\reals}{{\mathbb R}}
\newcommand{\norm}[1]{\left|\left|#1\right|\right|}
\newcommand{\unorm}[1]{{\left|\left|#1\right|\right|_u}}
\newcommand{\scriptR}{\mathcal{R}}
\newcommand{\scriptP}{\mathcal{P}}
\newcommand{\taggedP}{\dot{\mathcal{P}}}
\newcommand{\scriptQ}{\mathcal{Q}}
\newcommand{\taggedQ}{\dot{\mathcal{Q}}}


% Allows hyperlinks if compiled with pdflatex
\usepackage{hyperref}
\hypersetup{colorlinks}
\usepackage{color}
\definecolor{darkred}{rgb}{0.5,0,0}
\definecolor{darkgreen}{rgb}{0,0.5,0}
\definecolor{darkblue}{rgb}{0,0,0.5}
\hypersetup{ colorlinks,
                linkcolor=darkblue,
                filecolor=darkgreen,
                urlcolor=darkblue,
                citecolor=darkblue }
%hyperlink example is: \href{http://www.google.com}{google}

%add code!
\usepackage{listings}
\definecolor{mygreen}{rgb}{0,0.6,0}
\definecolor{mygray}{rgb}{0.5,0.5,0.5}
\definecolor{mymauve}{rgb}{0.58,0,0.82}
\lstset{ %
backgroundcolor=\color{white},   % choose the background color; you must add \usepackage{color} or \usepackage{xcolor}
basicstyle=\footnotesize,        % the size of the fonts that are used for the code
breakatwhitespace=false,         % sets if automatic breaks should only happen at whitespace
breaklines=true,                 % sets automatic line breaking
captionpos=b,                    % sets the caption-position to bottom
commentstyle=\color{mygreen},    % comment style
deletekeywords={...},            % if you want to delete keywords from the given language
escapeinside={\%*}{*)},          % if you want to add LaTeX within your code
extendedchars=true,              % lets you use non-ASCII characters; for 8-bits encodings only, does not work with UTF-8
frame=single,                    % adds a frame around the code
keepspaces=true,                 % keeps spaces in text, useful for keeping indentation of code (possibly needs columns=flexible)
keywordstyle=\color{blue},       % keyword style
language=Octave,                 % the language of the code
morekeywords={*,...},            % if you want to add more keywords to the set
numbers=left,                    % where to put the line-numbers; possible values are (none, left, right)
numbersep=5pt,                   % how far the line-numbers are from the code
numberstyle=\tiny\color{mygray}, % the style that is used for the line-numbers
rulecolor=\color{black},         % if not set, the frame-color may be changed on line-breaks within not-black text (e.g. comments (green here))
showspaces=false,                % show spaces everywhere adding particular underscores; it overrides 'showstringspaces'
showstringspaces=false,          % underline spaces within strings only
showtabs=false,                  % show tabs within strings adding particular underscores
stepnumber=2,                    % the step between two line-numbers. If it's 1, each line will be numbered
stringstyle=\color{mymauve},     % string literal style
tabsize=2,                       % sets default tabsize to 2 spaces
title=\lstname                   % show the filename of files included with \lstinputlisting; also try caption instead of title
}


\begin{document}
 
\setcounter{section}{15}
\section{Software reuse}
Reuse is promoted to see a greater return on software investment.
There may be:
Application system reuse, component reuse, object and function reuse.
Benefits include:
\begin{enumerate}
\item Increased dependability
\item Reduced process risk
\item Effective use of specialists
\item Standards compliance
\item Accelerated development
\end{enumerate}
Problems include:
\begin{enumerate}
\item Increased maintenance costs
\item Lack of total support
\item Not-invented-here syndrome
\item Creating, maintaining and using a component library
\item Finding, understanding and adapting reusable components.
\end{enumerate}
\subsection{The reuse landscape}
Key factors when considering planning reuse are:
\begin{enumerate}
\item The development schedule for the software
\item The expected software lifetime
\item The background, skill, and experience of the development team
\item The criticality of the software and its non-functional requirements
\item The application domain
\item The platform on which the system will run.
\end{enumerate}
Approaches that support software re-use:
\begin{center}
\begin{tabular}{l l}
Architectural patters & Design patterns\\
Component-based development & Application frameworks\\
Legacy-system wrapping & Service-oriented systems\\
Software product lines & COTS product reuse\\
ERP systems & Configurable vertical applications\\
Program libraries & Model-driven engineering\\
Program generators & Aspect-oriented software development
\end{tabular}
\end{center}
\subsection{Application frameworks}
Application frameworks are implemented as a collection of concrete and abstract obect classes in an object-oriented programming language.
Three classes of frameworks are:
\begin{enumerate}
\item System infrastructure frameworks
\item Middleware integration frameworks
\item Enterprise application frameworks
\end{enumerate}
Web application frameworks (WAFs) are also becomming very popular to create web pages.
\textbf{Frameworks are often implementations of design patterns}

\subsection{Software product lines}
Various types of specialization of a sotware product line may be developed:
\begin{enumerate}
\item Platform specialization
\item Environment specialization
\item Functional specialization
\item Process specialization
\end{enumerate}
Steps involved in extending a software product line are:
\begin{enumerate}
\item Elicit stakeholder requirements
\item Select the existing system that is the closest fit to the requirements
\item Renegotiate requirements
\item Adapt existing system
\item Deliver new family member
\end{enumerate}
\subsection{COTS product reuse}
A commercial-off-the-shelf product is a software system that can be adapted to the needs of different customers without changing the source code of the system.
There are two broad types of COTS product reuse: COTS-solution systems and COTS-integrated systems.
Solution systems consist of a generic application from a single vendor that is configured to customer requirements.
Integrated systems involve itegrating two or more COTS systems to create an application system.
\subsubsection{COTS-solution systems}
Solution systems are generic application systems that may be designed to support a particular business type.  On a larger scale, an Enterprise Resource Planning (ERP) system may support all of the manufacturing, ordering, and customer relationship management activities.  Key features of the ERP are:
\begin{enumerate}
\item A numver of modules to support different business functions.
\item Defined set of business processes associated with each module
\item A common dtabase that maintains information about all related business functions
\item Set of business rules that apply to all data in the database
\end{enumerate}

Extensive configuration may be involved.  Configuration activities include:
\begin{enumerate}
\item Selecting the required functionality from the system
\item Establishing a data model that defines how the organization's data will be structured in the system databse.
\item Defining business rules that paply to that data.
\item Defining the expected interactions with external systems.
\item Designing the input forms and the output reports generated by the system.
\item Desining new business processes that conform to the underlying process model supported by the system.
\end{enumerate}

\subsubsection{COTS-integrated systems}
Integrated systems are applications that include two or more COTS products or, sometimes, legacy applications sytems.  
Four problems with COTS integration:
\begin{enumerate}
\item Lack of control over functionality and performance.
\item Problems woth COTS system interoperability
\item No control over system evolution
\item Support form COTS vendors
\end{enumerate}
\end{document}

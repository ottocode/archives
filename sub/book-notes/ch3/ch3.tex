\documentclass{article}

\usepackage{fancyhdr}
\setlength{\headheight}{12pt}
\setlength{\textwidth}{17.2cm} \setlength{\textheight}{23cm}
\setlength{\topmargin}{-2.5cm} \setlength{\headsep}{1.6cm}
\setlength{\evensidemargin}{-.8cm}
\setlength{\oddsidemargin}{-.8cm}
\pagestyle{fancy}

%set-up page dimentions
\usepackage[top=1.5 in, bottom = 1 in ,left = 1 in, right = 1in]{geometry}

\setlength{\parskip}{12pt}  % 12 pt = space between paragraphs
\setlength{\parindent}{0pt} % 0 pt  = indentation
\usepackage{amsmath}
\usepackage{amssymb}
\usepackage{amsthm}
\usepackage{ifthen}
\usepackage{latexsym}
\usepackage{graphicx}
\usepackage{graphics}
\usepackage{psfrag}
\usepackage{graphpap}
\renewcommand{\P}{\text{P}}
\newcommand{\C}{\text{C}}


\newcommand{\natnums}{{\mathbb N}}
\newcommand{\algnums}{{\mathbb A}}
\newcommand{\rationals}{{\mathbb Q}}
\newcommand{\reals}{{\mathbb R}}
\newcommand{\norm}[1]{\left|\left|#1\right|\right|}
\newcommand{\unorm}[1]{{\left|\left|#1\right|\right|_u}}
\newcommand{\scriptR}{\mathcal{R}}
\newcommand{\scriptP}{\mathcal{P}}
\newcommand{\taggedP}{\dot{\mathcal{P}}}
\newcommand{\scriptQ}{\mathcal{Q}}
\newcommand{\taggedQ}{\dot{\mathcal{Q}}}


% Allows hyperlinks if compiled with pdflatex
\usepackage{hyperref}
\hypersetup{colorlinks}
\usepackage{color}
\definecolor{darkred}{rgb}{0.5,0,0}
\definecolor{darkgreen}{rgb}{0,0.5,0}
\definecolor{darkblue}{rgb}{0,0,0.5}
\hypersetup{ colorlinks,
                linkcolor=darkblue,
                filecolor=darkgreen,
                urlcolor=darkblue,
                citecolor=darkblue }
%hyperlink example is: \href{http://www.google.com}{google}

%add code!
\usepackage{listings}
\definecolor{mygreen}{rgb}{0,0.6,0}
\definecolor{mygray}{rgb}{0.5,0.5,0.5}
\definecolor{mymauve}{rgb}{0.58,0,0.82}
\lstset{ %
backgroundcolor=\color{white},   % choose the background color; you must add \usepackage{color} or \usepackage{xcolor}
basicstyle=\footnotesize,        % the size of the fonts that are used for the code
breakatwhitespace=false,         % sets if automatic breaks should only happen at whitespace
breaklines=true,                 % sets automatic line breaking
captionpos=b,                    % sets the caption-position to bottom
commentstyle=\color{mygreen},    % comment style
deletekeywords={...},            % if you want to delete keywords from the given language
escapeinside={\%*}{*)},          % if you want to add LaTeX within your code
extendedchars=true,              % lets you use non-ASCII characters; for 8-bits encodings only, does not work with UTF-8
frame=single,                    % adds a frame around the code
keepspaces=true,                 % keeps spaces in text, useful for keeping indentation of code (possibly needs columns=flexible)
keywordstyle=\color{blue},       % keyword style
language=Octave,                 % the language of the code
morekeywords={*,...},            % if you want to add more keywords to the set
numbers=left,                    % where to put the line-numbers; possible values are (none, left, right)
numbersep=5pt,                   % how far the line-numbers are from the code
numberstyle=\tiny\color{mygray}, % the style that is used for the line-numbers
rulecolor=\color{black},         % if not set, the frame-color may be changed on line-breaks within not-black text (e.g. comments (green here))
showspaces=false,                % show spaces everywhere adding particular underscores; it overrides 'showstringspaces'
showstringspaces=false,          % underline spaces within strings only
showtabs=false,                  % show tabs within strings adding particular underscores
stepnumber=2,                    % the step between two line-numbers. If it's 1, each line will be numbered
stringstyle=\color{mymauve},     % string literal style
tabsize=2,                       % sets default tabsize to 2 spaces
title=\lstname                   % show the filename of files included with \lstinputlisting; also try caption instead of title
}
\begin{document}

\setcounter{section}{2}
\section{Agile Software Development.}

Rapid-delivery of software is essential for some business environments.  

IBM introduced incremental development in 1980s (p57)

\textbf{Fundamental Characteristics of rapid software development:}
    \begin{enumerate}
        \item processes of specification, design, and implementation are interleaved.
        No detailed system specification, and design documentation is minimalized, or automatically generated
    \item The system is developed in a series of versions.
    \item UI developed often with interactive development that allows quick design/drawing
    \end{enumerate}

\subsection{Agile methods}

Agile methods intended to deliver software quickly to users.

\textbf{Agile Philosophy}
\begin{enumerate}
    \item Individuals and interactions over processes and tools
    \item Working software over comprehensive documentation
    \item Customer collaboration over contract negotiotion
    \item Responding to change over following a plan
    \item For the above, items on the left are valued more that on the right.
\end{enumerate}

\textbf{Successful areas for agile development}
\begin{enumerate}
    \item small or medium-sized products for sale.
    \item Custom system development within an organization, few external rules.
\end{enumerate}

Agile focus on small teams makes scaling more difficult.

\textbf{Diffuculties with agile development}
\begin{enumerate}
    \item Customer involvement is sometimes difficult to achieve.
    \item Team members may not have suitable perosnality.
    \item Prioritizing changes can be difficult
    \item Maintaining simplicity requires extra work.
    \item Challenges adapting culture.
    \item Writing contracts can be difficult for products.
\end{enumerate}

Two big questions: Are agile-developed systems maintainable, and can agile methods be effectively used for evolving a system in response to customer change requests.

Formal documentation in practice often not kept up to date.  Some agilers argue documentation is a waste of time.

\subsection{Plan-driven and agile development}


\subsection{Extreme programming (XP)}

So called for pushing iterative development to "extreme" levels.  

Requirements are expressed as scenarios (or "user stories").

Utilizes pair programming and tests written before actual code

\begin{enumerate}
    \item Incremental development supported through small, frequent releases of system.
    \item Continuous customer involvement
    \item Emphasis on people, not process, collective ownership of system code.
    \item Change embraced through regular system releases to customers and test-first development.
    \item Maintain simplicity by constant refactoring.
\end{enumerate}

"Spikes" used to design system architecture or develop system documentation.  No code is written.

Tries to fight against bloated system design by constant refactoring.

\textbf{Key features to testing in XP are:}
\begin{enumerate}
    \item test-first development
    \item incremental test development from scenarios
    \item user involvement in the thest development and validation
    \item use of automated testing frameworks
\end{enumerate}

Customer should have a role in the testing process to help develop acceptance tests for stories.
This is sometimes a major difficulty.

Testing is is incremental, just like the development.

\textbf{Reasons why test-first doesn't catch everything}
\begin{enumerate}
    \item Programmer prefer programming to testing and may take shortcuts.
    \item Certain tests are difficult to write incrementally.
    \item Difficult to judge completeness of a set of tests.
\end{enumerate}

\textbf{Pair Programming}

\textbf{Pair programming advantages}
\begin{enumerate}
    \item Supports collective ownership and responsibility for the system.
    \item Acts as informal review process.
    \item Helps support refactoring.
\end{enumerate}

\subsection{Agile project management}
Project management is often plan-drive.  Needs to be adapted for agile methods.

Scrum approach is agile that focuses on managing iterative development, instead of technical approaches to agile software engineering.

Three general phases
\begin{enumerate}
    \item Outline/planning phase
    \item Series of sprint cycles where each cycle develops an increment of the system.  Includes assess, select, develop, review.
    \item Project closure. 
\end{enumerate}

Key characteristics:
\begin{enumerate}
    \item Sprints are fixed lenght (2-4 weeks).  Correspond to development of a release of system in XP.
    \item Starting point for planning is the product backlog
    \item Selection phase involves all project team and customer to select features and functionality to develop.
    \item During development, team organizes and develops software.  Generally kept separate from customer.
    \item Review: present to stakeholders and start new sprint.
\end{enumerate}

"Stand-up" meetings to assess current progress.  Scrum designed with co-located teams in mind.

\subsection{Scaling agile methods}
\textbf{Large vs Small software development}
\begin{enumerate}
    \item Large systems are often collections of separate systems, difficult for one person to have grasp on entire project.
    \item Large systems are brownfield systems, interact with a number of existing systems.  Requirements are often inflexible.
    \item For larger systems, a significant amount of time is spent on system configuration, rather than original development.
    \item Large systems constrained by external rules.
    \item Longprocurement and development time for large systems.
    \item Diverse set of stakeholder on larger systems.
\end{enumerate}

Two different perspectives for scaling agile: "Scale Up," using agile to develop systems that cannot be done with a small team, vs "Scaling out,", how can agile methods be introduced across a large organization with lots of experience.

Some ways to use agile in large communities:
\begin{enumerate}
    \item Not possible to only focus on code.  More time needed to do up-front design and system documentation.
    \item Need to have cross-team communication handled.
    \item Continuous integration with the entire system.
\end{enumerate}


\end{document}

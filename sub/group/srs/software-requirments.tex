\documentclass{article}

\usepackage{fancyhdr}
\setlength{\headheight}{12pt}
\setlength{\textwidth}{17.2cm} \setlength{\textheight}{23cm}
\setlength{\topmargin}{-2.5cm} \setlength{\headsep}{1.6cm}
\setlength{\evensidemargin}{-.8cm}
\setlength{\oddsidemargin}{-.8cm}
%\pagestyle{fancy}

%set-up page dimentions
\usepackage[top=1 in, bottom = 1 in ,left = 1.5 in, right = 1.5in]{geometry}

\setlength{\parskip}{12pt}  % 12 pt = space between paragraphs
\setlength{\parindent}{12pt} % 0 pt  = indentation
\usepackage{amsmath}
\usepackage{amssymb}
\usepackage{amsthm}
\usepackage{ifthen}
\usepackage{latexsym}
\usepackage{graphicx}
\usepackage{graphics}
\usepackage{psfrag}
\usepackage{graphpap}
\renewcommand{\P}{\text{P}}
\newcommand{\C}{\text{C}}


\newcommand{\natnums}{{\mathbb N}}
\newcommand{\algnums}{{\mathbb A}}
\newcommand{\rationals}{{\mathbb Q}}
\newcommand{\reals}{{\mathbb R}}
\newcommand{\norm}[1]{\left|\left|#1\right|\right|}
\newcommand{\unorm}[1]{{\left|\left|#1\right|\right|_u}}
\newcommand{\scriptR}{\mathcal{R}}
\newcommand{\scriptP}{\mathcal{P}}
\newcommand{\taggedP}{\dot{\mathcal{P}}}
\newcommand{\scriptQ}{\mathcal{Q}}
\newcommand{\taggedQ}{\dot{\mathcal{Q}}}
\newcommand{\conman}{\texttt{ConMan }}


% Allows hyperlinks if compiled with pdflatex
\usepackage{hyperref}
\hypersetup{colorlinks}
\usepackage{color}
\definecolor{darkred}{rgb}{0.5,0,0}
\definecolor{darkgreen}{rgb}{0,0.5,0}
\definecolor{darkblue}{rgb}{0,0,0.5}
\hypersetup{ colorlinks,
                linkcolor=darkblue,
                filecolor=darkgreen,
                urlcolor=darkblue,
                citecolor=darkblue }
%hyperlink example is: \href{http://www.google.com}{google}

%add code!
\usepackage{listings}
\definecolor{mygreen}{rgb}{0,0.6,0}
\definecolor{mygray}{rgb}{0.5,0.5,0.5}
\definecolor{mymauve}{rgb}{0.58,0,0.82}
\lstset{ %
backgroundcolor=\color{white},   % choose the background color; you must add \usepackage{color} or \usepackage{xcolor}
basicstyle=\footnotesize,        % the size of the fonts that are used for the code
breakatwhitespace=false,         % sets if automatic breaks should only happen at whitespace
breaklines=true,                 % sets automatic line breaking
captionpos=b,                    % sets the caption-position to bottom
commentstyle=\color{mygreen},    % comment style
deletekeywords={...},            % if you want to delete keywords from the given language
escapeinside={\%*}{*)},          % if you want to add LaTeX within your code
extendedchars=true,              % lets you use non-ASCII characters; for 8-bits encodings only, does not work with UTF-8
frame=single,                    % adds a frame around the code
keepspaces=true,                 % keeps spaces in text, useful for keeping indentation of code (possibly needs columns=flexible)
keywordstyle=\color{blue},       % keyword style
language=Octave,                 % the language of the code
morekeywords={*,...},            % if you want to add more keywords to the set
numbers=left,                    % where to put the line-numbers; possible values are (none, left, right)
numbersep=5pt,                   % how far the line-numbers are from the code
numberstyle=\tiny\color{mygray}, % the style that is used for the line-numbers
rulecolor=\color{black},         % if not set, the frame-color may be changed on line-breaks within not-black text (e.g. comments (green here))
showspaces=false,                % show spaces everywhere adding particular underscores; it overrides 'showstringspaces'
showstringspaces=false,          % underline spaces within strings only
showtabs=false,                  % show tabs within strings adding particular underscores
stepnumber=2,                    % the step between two line-numbers. If it's 1, each line will be numbered
stringstyle=\color{mymauve},     % string literal style
tabsize=2,                       % sets default tabsize to 2 spaces
title=\lstname                   % show the filename of files included with \lstinputlisting; also try caption instead of title
}


\begin{document}
\begin{titlepage}
\Huge
\vspace{2cm}
\begin{center}
Software Requirements Specification for the \texttt{ConMan}\\
\vspace{3cm}
\Large
Authors:\\ 

\begin{tabular}{l l}\hline
Angel De Castro & Luis Retana\\
Nicholas Otto   & Christopher Yip
\end{tabular}
\vspace{1cm}

Version 0.1\\
10/18/2013
\end{center}


\end{titlepage}

\normalsize

\newpage
\tableofcontents

\newpage
\section{Introduction}
\subsection{Purpose}
%Identify the product whose software requirements are specified in this document, including the revision or release number. Describe the scope of the product that is covered by this SRS document, particularly if this SRS document describes only part of the system or a single subsystem.
The Calendar, Organization, and Notification Manager - \texttt{ConMan} - is a web-suite that gives project managers the tools needed to plan, assign, and follow-up on tasking given to project team members.  This SRS document covers all aspects of \texttt{ConMan}.

%\subsection{Document Conventions}
%Describe any standards or typographical conventions that were followed when writing this SRS document, such as fonts or highlighting that have special significance. For example, state whether priorities  for higher-level requirements are assumed to be inherited by detailed requirements, or whether every requirement statement is to have its own priority.

%Todo or takeout

\subsection{Intended Audience} % and Reading Suggestions
%<Describe the different types of reader that the document is intended for, such as developers, project managers, marketing staff, users, testers, and documentation writers. Describe what the rest of this SRS document contains and how it is organized. Suggest a sequence for reading the document, beginning with the overview sections and proceeding through the sections that are most pertinent to each reader type.>
This document is intended primarily for developers and reviewers of the \texttt{ConMan} suite.  
Advanced users of \texttt{ConMan} may find reading this SRS document interesting reading as well.

\subsection{Product Scope}
%<Provide a short description of the software being specified and its purpose, including relevant benefits, objectives, and goals. Relate the software to corporate goals or business strategies. If a separate vision and scope document is available, refer to it rather than duplicating its contents here.>
\texttt{ConMan}'s main objective is to give project managers and project team members a simple, but effective, tool for planning their tasking and keeping to a schedule.  
\texttt{ConMan} accomplishes its purpose with these mechanisms:
\begin{itemize}
    \item Tight integration of scheduled tasking with a calendar view.  \texttt{ConMan} will allow a user to see at a glance the tasks they are assigned, how much progress has already been made, and the effort remaining to complete the task.
    \item Meaningful notifications and updates.  Users can choose to select a notification schedule themselves or recieve automatic notifications of changes to their tasks and upcoming deadlines.
\end{itemize}

%\subsection{References}
%<List any other documents or Web addresses to which this SRS document refers. These may include user interface style guides, contracts, standards, system requirements specifications, use case documents, or a vision and scope document. Provide enough information so that the reader could access a copy of each reference, including title, author, version number, date, and source or location.>

%TODO 

\newpage
\section{Overall Description}
\subsection{Product Perspective}
\texttt{ConMan} is inspired by other attempts to create a useful calendar, checklist, and task management software suite such as Trello, SourceForge, and some basic calendar applications for iPhone and Android.  \texttt{ConMan's} approach to creating a task-management suite is emphasizing simplicity and ease of use as the primary development and design philosophies.

\subsection{User Classes and Characteristics}
The two user classes for \texttt{ConMan} are managers and members.  
Managers have privileged commands available which effect an entire team.
Members are assigned to teams and can modify the tasks that are assigned to them.
When a member switches to the individual context, each member becomes his/her own manager.


\subsection{Product Functions}
\texttt{ConMan's} functions can be divided into two categories: application functions and user functions.  Application functions are functions that describe the basic functionality of \texttt{ConMan}.  
They include:
\begin{enumerate}
    \item Creating a user profile.
    \item Switching to calendar view.
    \item Switching to task view.
    \item Switching to checklist view.
    \item Receiving notifications and updates.
\end{enumerate}

User functions are further divided into manager and member functions.
Manager functions are only available to users designated as project managers. Users are designated as project managers during 
the creation of their team. 
Manager functions include:
\begin{enumerate}
    \item Creating and deleting teams.
    \item Adding and modifying tasks for a team.
    \item Creating, modifying, and deleting checklists for the team.
    \item Creating group notifications.
    \item Performing member verification functions.
\end{enumerate}
Finally, the member functions apply to every user of \texttt{ConMan}.  These functions include:
\begin{enumerate}
    \item Writing task progress notes.
    \item Writing checklist progress notes.
    \item Switching between team and individual views.
    \item Marking checklist items as complete (if the checklist is assigned to the member).
\end{enumerate}

\subsection{Operating Environment}
%<Describe the environment in which the software will operate, including the hardware platform, operating system and versions, and any other software components or applications with which it must peacefully coexist.>
\texttt{ConMan} deploys on an ASP.NET server using a SQL Server Express database.
The target browsers in which \texttt{ConMan} will run are Firefox and Chrome. The application's 
client-side code (JavaScript, jQuery, etc.) will run within these browsers.


%\subsection{Design and Implementation Constraints}
%<Describe any items or issues that will limit the options available to the developers. These might include: corporate or regulatory policies; hardware limitations (timing requirements, memory requirements); interfaces to other applications; specific technologies, tools, and databases to be used; parallel operations; language requirements; communications protocols; security considerations; design conventions or programming standards (for example, if the customer’s organization will be responsible for maintaining the delivered software).>
%TODO fill in as we get better idea.

\subsection{User Documentation}
%<List the user documentation components (such as user manuals, on-line help, and tutorials) that will be delivered along with the software. Identify any known user documentation delivery formats or standards.>
The documents available describing \texttt{ConMan} include this software requirements specification, a 
user manual, design document, testing report document and
an in-app help menu.

All of the documentation will be available in PDF.

%\subsection{Assumptions and Dependencies}
%<List any assumed factors (as opposed to known facts) that could affect the requirements stated in the SRS document. These could include third-party or commercial components that you plan to use, issues around the development or operating environment, or constraints. The project could be affected if these assumptions are incorrect, are not shared, or change. Also identify any dependencies the project has on external factors, such as software components that you intend to reuse from another project, unless they are already documented elsewhere (for example, in the vision and scope document or the project plan).>

\newpage
\section{External Interface Requirements}
%%%%BREAK
\subsection{User Interfaces}
User Interface characteristics for the \texttt{ConMan} web application are as follows:

\begin{itemize}
    \item The application will use standard HTML, CSS, JavaScript, and possibly ASP.NET controls to structure and create the user interface.
    \item The web application will provide a consistent layout through the use of consistent coloring, formatting, etc.
    \item The web application will provide the user with a "Help" button so that the user can quickly and easily learn how to use each component of the application.
    \item The web user interface will handle errors in different ways:
	\begin{itemize}
		\item If a web form is not completed in a valid manner, then the UI will present an error message in red text to the right of each form field that was not valid. 
			Each of these error messages will display a message that is appropriate for the error. 
			For example, if a field accepts non-negative numbers and the user enters -1.5, the UI will display an error message indicating that input must not be negative.
		\item All other errors will be presented to the user via a dialog/message box that cannot be ignored without acknowledgment (i.e. pressing "OK"). 
			This message must present the user with helpful messages that describe the error. 
			For example, an error message for a database transaction may state "ERROR: Communication with the server's database has been lost".
	\end{itemize}
    \item The web application will allow the user to quickly switch between calendar, checklist, and calendar/checklist views.
	\begin{itemize}
		\item Calendar View
		\begin{itemize}
			\item The calendar view shows the current month's calendar and for each day, it will display the name of each task that has a checklist due on that day. 
					By clicking on the name of a task, the web application will redirect the user to the task view (discussed below under the "Task View" bullet).
		\end{itemize}
		
		\item Task View
		\begin{itemize}
			\item The task view will show all information about the selected task. The task name, description, checklist(s), and notes will be available to the
					user. The user will also be able to edit (add a checklist or modify other task information) and delete a task. In addition, this view will enable the
					user to sign up for notifications concerning the current task. Clicking on a checklist from this view will redirect the user to the checklist view (discussed below 
					under the "Checklist View" bullet).
		\end{itemize}
		
		\item Checklist View
		\begin{itemize}
			\item The checklist view will show all information about the selected checklist. The checklist name, description, parent task, item(s), and notes will be available to the
					user. The user will also be able to edit (add an item to the checklist or modify other checklist information) and delete a checklist. In addition, this view will enable the
					user to sign up for notifications concerning the current checklist. Clicking on a checklist item will display the details associated with that checklist item.
		\end{itemize}
		
	\end{itemize}
    

	\item The web application will have log-in information available in the top-right corner of the page.
		\begin{itemize}
			\item If the user is not logged in, the top-right corner of the page will prompt the user for log-in credentials
			\item If the user is logged in, the top-right corner of the page will display the users log-in name and will display a link/button for the user to sign out
		\end{itemize}
	\item The web application will alert the user that one or more checklists are due for the current day by presenting the user with dialog/message box notifications stating which tasks have checklists due. Notifications
			will also be presented for checklists with due dates in the near future (e.g., due dates within the next 3 days).
	\item The web application will also alert the user via notifications if any tasks or checklists assigned to them have been modified.
\end{itemize}

\subsection{Hardware Interfaces}
   %<Describe the logical and physical characteristics of each interface between the software product and the hardware components of the system. This may include the supported device types, the nature of the data and control interactions between the software and the hardware, and communication protocols to be used.>
Not applicable for \texttt{ConMan}.

\subsection{Software Interfaces}
      %<Describe the connections between this product and other specific software components (name and version), including databases, operating systems, tools, libraries, and integrated commercial components. Identify the data items or messages coming into the system and going out and describe the purpose of each. Describe the services needed and the nature of communications. Refer to documents that describe detailed application programming interface protocols. Identify data that will be shared across software components. If the data sharing mechanism must be implemented in a specific way (for example, use of a global data area in a multitasking operating system), specify this as an implementation constraint.>

As mentioned in section 3.1, the web application will be built using HTML, CSS, JavaScript, and ASP.NET. 
The application will also need to interface with the database located on the server through the use of SQL and database libraries such as those provided by the .NET framework. 
This database will store all data pertaining to user information (e.g., ID, name, log-in credentials, age, employer, team name, email address, etc.), company information (e.g., ID, name, employees, teams, etc.), task information (e.g., ID, due date, creator team ID, employees, checklists, etc.), and checklist information (e.g., ID, creator ID, due date, task ID, employees IDs, list, etc.).
        
\subsection{Communications Interfaces}
                         %<Describe the requirements associated with any communications functions required by this product, including e-mail, web browser, network server communications protocols, electronic forms, and so on. Define any pertinent message formatting. Identify any communication standards that will be used, such as FTP or HTTP. Specify any communication security or encryption issues, data transfer rates, and synchronization mechanisms.>
The web application will rely on the following technologies for communication:
\begin{itemize}
    \item SMTP for sending out email notifications to users concerning checklists with upcoming due dates
    \item SQL for performing transactions with the application's database
\end{itemize}

\newpage
\section{System Features}
The system will have a simple layout.
The main screen will show a simple month-view calendar with the current day highlighted. 
The name of an assigned task or checklist will appear on the day of the month when it is due. 
The calendar then offers the possibility of clicking on any day or task.
If a task is clicked, a list will appear with the task details along with options for adding, deleting or editing checklists associated with the task. 
If the day of the month was clicked, the only option will be to add a new task. 
The interface will also have a menu to select user context, change view and any other secondary features helpful to the system (e.g. help-section links).

\subsection{Calendar View}
\subsubsection{Description and Priority}
Creating a calendar view will be the main priority.
It should allow the user to click on any day to see or create new tasks.  
The system should also implement the possibility of changing months to see past and future tasks. 
This will be of the highest priority, 9, since all future features are based on the user being able to choose a day from the calendar.

\subsubsection{Stimulus/Response Sequences}
After logging into \texttt{ConMan} the initial screen will be the calendar view.  When the user clicks on a day it should transition into a to-do list interface. 
Here the user can see, add, edit or delete tasks. 
The view feature will show a form with the tasks information.

\subsubsection{Functional Requirements}
The interface shall be intuitive, with minimal training required to learn how it is used. 
To minimize user error, all action buttons will be large and simple to understand.

\subsection{Adding, Editing, Deleting Tasks}
\subsubsection{Description and Priority}
Modifying a task is the primary user action in \texttt{ConMan}. 
Adding a task will create a new entry for a specific day inside the calendar view. Deleting a task completely removes a task, while editing a task will allow the user to change any of a task fields. 
This will be of the highest priority, 9, since all of the system functionality depends on the user being able to assign tasks to a day.
\subsubsection{Stimulus/Response Sequences}
Once the day from the calendar view is selected, the user will be able to choose between adding, deleting, editing and viewing the task. 
Adding a task will take the user to a form where he can enter the initial task values and then create the task. 
Editing will take the user to the same form with the fields already completed and deleting will remove the task entry. 
\subsubsection{Functional Requirements}
The task form will require the usual fields in a calendar. 
It should include date, time, description, and alarm. 
Once filled the system adds the task to the day specified. 
The form will check for errors in the information supplied before accepting the task. 
If the user modifies the deadline date, the change should also be reflected on the calendar and a notification should be sent to all employees associated with the task.

\subsection{Adding, Editing, Deleting Checklists}
\subsubsection{Description and Priority} 
Adding a checklist to a task will update the task by placing the new checklist in the task's checklist list. 
Deleting a checklist completely removes a checklist, while editing a checklist will allow the user to change any of a checklist fields. 
This will be of priority, 7, since the usefulness of tasks are greatly diminished if checklists cannot be associated with the tasks.
\subsubsection{Stimulus/Response Sequences}
Once a checklist is selected, the user will be able to choose between adding, deleting, editing and viewing the checklist. 
Adding a checklist will take the user to a form were he can enter the initial checklist values and then create the checklist. 
Editing will take you to the same form as adding a checklist, but with the fields completed with existing information and deleting will remove the checklist and all references to it. 
\subsubsection{Functional Requirements}
The checklist form will require the usual fields in a calendar. 
It should include date, time, description, and alarm. 
Once the form is filled, the system then adds the checklist to the specified task. 
The form will check for errors in the information supplied before accepting the checklist. 
If the user changes the date this should also be reflected on the calendar and a notification should be sent out to all employees associated with the task.

\subsection{User Profiles}
\subsubsection{Description and Priority}
This feature allows a user to log into a profile, loading all the tasks previously created onto the calendar view. 
A simple authentication process is required to ensure privacy; a user will also be able to create an account on startup. 
Since this feature is important but not critical to system use, User Profiles have a priority of 7.

\subsubsection{Stimulus/Response Sequences}
Users will see an authentication screen each time the log into \texttt{ConMan}. 
The authentication screen will offer the possibility of creating a new user or entering existing credentials. 
If a new user is created, the system will transition into the user creation interface where she can complete the fields. 
If the user already has an account, she will be showed her calendar with all the previously created tasks. 
\subsubsection{Functional Requirements}
The new user form will require the usual fields: name, surname, password, email, etc. 
After the fields are supplied, the system will take the user to the calendar. 
The authentication process must be secure. 
Therefore, it must control who accesses each profile since data is confidential. 
If incorrect user credentials are entered or if the information for the new user is incorrect, the user will be prompted with a error message.

\subsection{Notification System}
\subsubsection{Description and Priority}
This feature can be activated at the task creation form. 
If activated, this feature will notify the user with an email and with a message/dialog box (via the web application) when a task has an upcoming due date. The notification alerts will have priority 5.

\subsubsection{Stimulus/Response Sequences}
This option will be made available to the user once she has entered the task creation form. 
It will give the user the option to choose a time to remind her and possibly other users of the event. 
 
\subsubsection{Functional Requirements}
The system will have to validate the time entered and ensure that it is plausible and formatted correctly.  If an invalid notification time is set, the user will be informed of the error. 

\newpage
\section{Other Nonfunctional Requirements}
\subsection{Performance Requirements}
%<If there are performance requirements for the product under various circumstances, state them here and explain their rationale, to help the developers understand the intent and make suitable design choices. Specify the timing relationships for real time systems. Make such requirements as specific as possible. You may need to state performance requirements for individual functional requirements or features.>
As a general requirement, the user should be able to visually see some type of response from the program within 0.2 seconds after a valid user action.  
If, for instance, a user performs an action that requires validation from the server (such as authenticating the user upon login), this requirement does not stipulate that the server will finish processing the request within 0.2 seconds, but rather the program will give at least a loading indicator to signal the user that the action was been received.

All e-mails sent by \texttt{ConMan} will be sent within 5 minutes of the indicated alarm time.
\texttt{ConMan}, of course, has no control over the performance of the user's email network which will receive the mail.


\subsection{Safety Requirements}
%<Specify those requirements that are concerned with possible loss, damage, or harm that could result from the use of the product. Define any safeguards or actions that must be taken, as well as actions that must be prevented. Refer to any external policies or regulations that state safety issues that affect the product’s design or use. Define any safety certifications that must be satisfied.>
There are no known physical safety requirements at this time.  Caution should be exercised when using any unfamiliar device, however.\\

\subsection{Security Requirements}
%<Specify any requirements regarding security or privacy issues surrounding use of the product or protection of the data used or created by the product. Define any user identity authentication requirements. Refer to any external policies or regulations containing security issues that affect the product. Define any security or privacy certifications that must be satisfied.>
A user should only be able to log onto \texttt{ConMan} with valid user credentials.
Furthermore, a user should only be able to modify tasks or checklists which they are assigned to.
Also, member level users can only view tasks assigned to the team to which they belong.

Manager level users, however, are the only user class that can add or assign tasks to teams.  Also, Manager users are solely responsible for marking tasks finished  or checklists as complete.  Basic users will be allowed to perform these functions.

\subsection{Software Quality Attributes}
%<Specify any additional quality characteristics for the product that will be important to either the customers or the developers. Some to consider are: adaptability, availability, correctness, flexibility, interoperability, maintainability, portability, reliability, reusability, robustness, testability, and usability. Write these to be specific, quantitative, and verifiable when possible. At the least, clarify the relative preferences for various attributes, such as ease of use over ease of learning.>
The source code will adhere to the following attributes:
\begin{itemize}
    \item The source code will be reasonably documented.  All non-trivial functions or methods will have header documentation describing the following:
        \begin{enumerate}
            \item Name
            \item General description, including pre-conditions
            \item List and description of input parameters
            \item List and description of returned values
            \item Information about state change during method execution (post-condition)
            \item Functions which are entirely described by the function name or private functions specific to a small code block need not satisfy all of these requirements.
        \end{enumerate}
    \item The source code will adhere to a single programming style, namely the Stroustrup variant of K\&R style.
    \item All code blocks will be enclosed in curly brackets, even if not explicitly required.
\end{itemize}

Above all else, the system should be at least minimally usable upon delivery.
If time allows, features which increase user flexibility will be added, but attempts to increase user options must not delay the project delivery.

\subsection{Business Rules}
%<List any operating principles about the product, such as which individuals or roles can perform which functions under specific circumstances. These are not functional requirements in themselves, but they may imply certain functional requirements to enforce the rules.>
This product will ship by the end of the fall semester.


\newpage
%\section{Other Requirements}
%<Define any other requirements not covered elsewhere in the SRS document. This might include database requirements, internationalization requirements, legal requirements, reuse objectives for the project, and so on. Add any new sections that are pertinent to the project.>


%\section{Appendix A: Glossary}
%<Define all the terms necessary to properly interpret the SRS document, including acronyms and abbreviations. You may wish to build a separate glossary that spans multiple projects or the entire organization, and just include terms specific to a single project in each SRS document.>


%\section{Appendix B: Analysis Models}
%<Optionally, include any pertinent analysis models, such as data flow diagrams, class diagrams, state-transition diagrams, or entity-relationship diagrams.>


%\section{Appendix C: To Be Determined List}
%<Collect a numbered list of the TBD (to be determined) references that remain in the SRS document so they can be tracked to closure.>

\end{document}

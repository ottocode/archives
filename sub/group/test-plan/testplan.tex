\documentclass{article}

\usepackage{fancyhdr}
\setlength{\headheight}{12pt}
\setlength{\textwidth}{17.2cm} \setlength{\textheight}{23cm}
\setlength{\topmargin}{-2.5cm} \setlength{\headsep}{1.6cm}
\setlength{\evensidemargin}{-.8cm}
\setlength{\oddsidemargin}{-.8cm}
%\pagestyle{fancy}

%set-up page dimentions
\usepackage[top=1 in, bottom = 1 in ,left = 1.5 in, right = 1.5in]{geometry}

\setlength{\parskip}{12pt}  % 12 pt = space between paragraphs
\setlength{\parindent}{12pt} % 0 pt  = indentation
\usepackage{amsmath}
\usepackage{amssymb}
\usepackage{amsthm}
\usepackage{ifthen}
\usepackage{latexsym}
\usepackage{graphicx}
\usepackage{graphics}
\usepackage{psfrag}
\usepackage{graphpap}
\renewcommand{\P}{\text{P}}
\newcommand{\C}{\text{C}}


\newcommand{\natnums}{{\mathbb N}}
\newcommand{\algnums}{{\mathbb A}}
\newcommand{\rationals}{{\mathbb Q}}
\newcommand{\reals}{{\mathbb R}}
\newcommand{\norm}[1]{\left|\left|#1\right|\right|}
\newcommand{\unorm}[1]{{\left|\left|#1\right|\right|_u}}
\newcommand{\scriptR}{\mathcal{R}}
\newcommand{\scriptP}{\mathcal{P}}
\newcommand{\taggedP}{\dot{\mathcal{P}}}
\newcommand{\scriptQ}{\mathcal{Q}}
\newcommand{\taggedQ}{\dot{\mathcal{Q}}}
\newcommand{\conman}{\texttt{ConMan }}


% Allows hyperlinks if compiled with pdflatex
\usepackage{hyperref}
\hypersetup{colorlinks}
\usepackage{color}
\definecolor{darkred}{rgb}{0.5,0,0}
\definecolor{darkgreen}{rgb}{0,0.5,0}
\definecolor{darkblue}{rgb}{0,0,0.5}
\hypersetup{ colorlinks,
                linkcolor=darkblue,
                filecolor=darkgreen,
                urlcolor=darkblue,
                citecolor=darkblue }
%hyperlink example is: \href{http://www.google.com}{google}

%add code!
\usepackage{listings}
\definecolor{mygreen}{rgb}{0,0.6,0}
\definecolor{mygray}{rgb}{0.5,0.5,0.5}
\definecolor{mymauve}{rgb}{0.58,0,0.82}
\lstset{ %
backgroundcolor=\color{white},   % choose the background color; you must add \usepackage{color} or \usepackage{xcolor}
basicstyle=\footnotesize,        % the size of the fonts that are used for the code
breakatwhitespace=false,         % sets if automatic breaks should only happen at whitespace
breaklines=true,                 % sets automatic line breaking
captionpos=b,                    % sets the caption-position to bottom
commentstyle=\color{mygreen},    % comment style
deletekeywords={...},            % if you want to delete keywords from the given language
escapeinside={\%*}{*)},          % if you want to add LaTeX within your code
extendedchars=true,              % lets you use non-ASCII characters; for 8-bits encodings only, does not work with UTF-8
frame=single,                    % adds a frame around the code
keepspaces=true,                 % keeps spaces in text, useful for keeping indentation of code (possibly needs columns=flexible)
keywordstyle=\color{blue},       % keyword style
language=Octave,                 % the language of the code
morekeywords={*,...},            % if you want to add more keywords to the set
numbers=left,                    % where to put the line-numbers; possible values are (none, left, right)
numbersep=5pt,                   % how far the line-numbers are from the code
numberstyle=\tiny\color{mygray}, % the style that is used for the line-numbers
rulecolor=\color{black},         % if not set, the frame-color may be changed on line-breaks within not-black text (e.g. comments (green here))
showspaces=false,                % show spaces everywhere adding particular underscores; it overrides 'showstringspaces'
showstringspaces=false,          % underline spaces within strings only
showtabs=false,                  % show tabs within strings adding particular underscores
stepnumber=2,                    % the step between two line-numbers. If it's 1, each line will be numbered
stringstyle=\color{mymauve},     % string literal style
tabsize=2,                       % sets default tabsize to 2 spaces
title=\lstname                   % show the filename of files included with \lstinputlisting; also try caption instead of title
}


\begin{document}
\begin{titlepage}
\Huge
\vspace{2cm}
\begin{center}
Change Request for the \texttt{ConMan} Application\\
\end{center}
\vspace{2cm}
\Large
To:\\
\texttt{ConMan} Management\\

From:\\ 
Development Team

\begin{tabular}{l l}
Angel De Castro & Luis Retana\\
Nicholas Otto   & Christopher Yip
\end{tabular}
\vspace{1cm}

\begin{center}
12/14/2013
\end{center}


\end{titlepage}

\normalsize

\newpage
\tableofcontents

\newpage

\section{Introduction}
\subsection{Purpose of the Test Plan Document}
The Test Plan document contains all information about various tests and 
how to implement various tests. The document's intended audience is the 
\texttt{ConMan} project manager, development team, and testing team.

\newpage
\section{Compatibility Testing}
\subsection{Test Risks / Issues}
Due to the nature of \texttt{ConMan}, no risks will manifest during compatibility 
testing. 

\subsection{Items to be tested / not tested}
All ASP.NET web forms (i.e. files ending in .aspx) shall be tested in both 
Mozilla Firefox and Google Chrome web browsers. No other web browsers will 
be used for compatibility testing as \texttt{ConMan} is built to work for Firefox 
and Chrome.\newline

Tests:
\begin{itemize}
\item Item to test: CMP-TEST-1
\item Test Description: Test every web page in \texttt{ConMan} using Google Chrome
\item Test Responsibility: Luis Retana
\end{itemize}
\begin{itemize}
\item Item to test: CMP-TEST-2
\item Test Description: Test every web page in \texttt{ConMan} using Mozilla Firefox
\item Test Responsibility: Nick Otto
\end{itemize}

\subsection{Test Approach(es)}
CMP-TEST-1 and CMP-TEST-2:
\begin{itemize}
\item Prior to beginning tests, cookies must be enabled on both Firefox and Chrome 
to allow \texttt{ConMan} to function properly.\newline \newline
Both tests must be performed at the same time and at the same location. 
The people responsible for each test shall perform the tests on computers that 
are next to each other. Also, both tests must be testing the same page at any given 
time. For example, if the people responsible for CMP-TEST-1 are testing the 
TeamPage.aspx web page in Chrome, the people responsible for CMP-TEST-2 should test 
TeampPage.aspx in Frefox.\newline \newline
	These approaches intend to facilitate the behavior of \texttt{ConMan} across 
different browsers by allowing both testing parties to compare how the behavior 
of \texttt{ConMan} changes (if at all) when used on different broswers.
\end{itemize}


\subsection{Test Pass / Fail Criteria}
CMP-TEST-1 and CMP-TEST-2:
\begin{itemize}
\item Both Firefox and Chrome browsers must display all user interface elements 
in accordance to how they are depicted in the Visual Studio Web Form Designer. This 
also means that there should be no differences in how the user interface is displayed 
in the two web browsers.
\item All of a web pages functions (e.g., creating a new user, updating a team, etc.) must 
function correctly.
\item Intra-site navigation must correctly navigate to the desired screen.
\end{itemize}

\subsection{Test Entry / Exit Criteria}
Testing shall commence with every major and minor release of \texttt{ConMan}. Testing 
will cease when all pages are tested within Chrome and Firefox.

\subsection{Test Deliverables}
A report shall be delivered stating whether the tests passed or failed. If tests 
failed, a SCR (Software Change Request) shall be created to allow the development 
team to correct the issues. In addition, the report should include a list of any 
SCRs generated for tests.

\subsection{Test Suspension / Resumption Criteria}
Testing must be suspended if it is determined that the web broswers or computers 
used in testing are not functioning properly. Testing can resume once the web 
broswers and computers are determined to be fit for testing. 

\subsection{Test Environmental / Staffing / Training Needs}
These tests require computers with Firefox and Chrome installed. These computers 
must have access to \texttt{ConMan} on the local machine or on a test server. People 
responsible for carrying out the tests must have knowledge of the user interface 
requirements and the functionality that must be supported for each web page.

\newpage
\section{Conformance Testing}
\subsection{Test Risks / Issues}
k

\subsection{Items to be Tested / Not Tested}
l 

\subsection{Test Approach(s)}
m 

\subsection{Test Regulatory / Mandate Criteria}
n

\subsection{Test Pass / Fail Criteria}
o

\subsection{Test Entry / Exit Criteria}
p

\subsection{Test Deliverables}
q

\subsection{Test Suspension / Resumption Criteria}
r

\subsection{Test Environmental / Staffing / Training Needs}
s

\newpage
\section{Performance and Load Testing}
\subsection{Test Risks / Issues}
Due to the nature of \texttt{ConMan}, no risks will manifest during peformance 
and load testing. The sole issue concerns how to test the performance of \texttt{ConMan} 
under an average load (e.g., 30 simutaneous users) with a small development and testing 
team. To resolve this issue, performance and load testing will utilize "outside" 
participants (i.e., people that are not affiliated with the 
\texttt{ConMan} team) who will be instructed on how to carry out the tests.

\subsection{Items to be Tested / Not Tested}
Each page that performs queries and transactions with the \texttt{ConMan} 
database shall be tested.

Tests:
\begin{itemize}
\item Item to test: PL-TEST-1
\item Test Description: Test each database function while that same function is being 
executed by all other testers. Ensure the database action's total elapsed time is 
five seconds or less. 
\item Test Responsibility: Group of 30 "outside" testers
\end{itemize}
\begin{itemize}
\item Item to test: PL-TEST-2
\item Test Description: Test each database function while that same function is being 
executed by all other testers. Ensure the database action yields the expected results. 
\item Test Responsibility: Group of 30 "outside" testers
\end{itemize}

\subsection{Test Approach(s)}
PL-TEST-1 and PL-TEST-2:
\begin{itemize}
\item Prior to testing, the testing team must gather 30 "outside" participants 
and teach them how to perform tests and how to judge whether \texttt{ConMan} passes 
or fails. During testing, a coordinator must instruct the testers to attempt to 
perform a specific action that requires communication with the \texttt{ConMan} database. 
Total elapsed time and result accuracy for each database action shall be recorded so 
that \texttt{ConMan}'s performance under expected user load can be judged as passing 
or failing.
\end{itemize}

\subsection{Test Pass / Fail Criteria}
PL-TEST-1:
\begin{itemize}
\item All database queries and transactions must have a total elapsed time of five 
seconds or less.
\end{itemize}
PL-TEST-2:
\begin{itemize}
\item All database queries and transactions must produce the expected results. For example, 
querying the database for all of a user's tasks should return all tasks's assigned to 
the user and no tasks that are not assigned to the user.
\end{itemize}

\subsection{Test Entry / Exit Criteria}
Testing shall begin with every major change to the database schema and/or to database SELECT, 
INSERT, DELETE, and UPDATE commands. Testing finishes once all database actions have been 
tested. 

\subsection{Test Deliverables}
A report shall be delivered stating whether the tests passed or failed. The report 
must also include information about which database action(s) failed and why it failed (e.g., 
query or transaction results were incorrect or total elapsed time was too long. If tests 
failed, a SCR (Software Change Request) shall be created to allow the development 
team to correct the issues. In addition, the report should include a list of any 
SCRs generated for tests.

\subsection{Test Suspension / Resumption Criteria}
Testing must be suspended if 30 "outside" testers are not available throughout the 
testing. Once a total of 30 testers are grouped together, testing may resume.

\subsection{Test Environmental / Staffing / Training Needs}
These tests require computers with Firefox and Chrome installed. These computers must
have access to ConMan on the local machine or on a test server. People responsible for
carrying out the tests must have knowledge of how to record that total elasped time 
of a database action and how to judge whether or not the results of that action are 
correct or not.

\newpage
\section{Stress Testing}
\subsection{Test Risks / Issues}
d

\subsection{Items to be Tested / Not Tested}
e

\subsection{Test Approach(s)}
f

\subsection{Test Regulatory / Mandate Criteria}
g

\subsection{Test Pass / Fail Criteria}
h

\subsection{Test Entry / Exit Criteria}
i

\subsection{Test Deliverables}
j

\subsection{Test Suspension / Resumption Criteria}
k

\subsection{Test Environmental / Staffing / Training Needs}
l

\section{System Testing}
There is no need for system testing as \texttt{ConMan} is developed on top 
of the .NET framework and web browsers.

\newpage
\section{Unit Testing}
\subsection{Test Risks / Issues}
No risks and issues are associated with the execution of the security tests.

\subsection{Items to be Tested / Not Tested}
The following items will be tested:
\begin{enumerate}
\item Account Creation
\item Account Login
\item Team Creation
\item Add Team Members
\item Add Tasks
\end{enumerate}

\subsection{Test Approach(s)}
For each of the test cases, a variety of incorrect inputs will be given to ensure the system correctly handles those cases.


\subsection{Testing Cases and Expected Results}
\subsubsection{Account Creation}
\textbf{Motivation:} A user decides that \texttt{ConMan}
sounds really cool and decides to give it a try. Unfortunately this user has difficulty typing. 
After navigating to the account
creation page, the user tries to create an account.

Potential cases:
\begin{enumerate}
    \item E-mail field is incorrectly entered.
        \begin{itemize}
            \item Input: An invalid e-mail (a description of a valid e-mail
will be determined later).
            \item Output: An error message will be displayed to the user
prompting them to enter a valid e-mail address.
        \end{itemize}
    \item Account creation fields are left blank.
        \begin{itemize}
            \item Input: Any of the fields for first name, last name, e-mail,
or password are left blank.
            \item Output: An error message appears prompting the user to
populate all of the fields.
        \end{itemize}
    \item An e-mail already in use has been entered.
        \begin{itemize}
            \item Input: An e-mail already in the \texttt{ConMan} database.
            \item Output: An error message prompting the user to either enter
a different e-mail or log onto existing account.
        \end{itemize}
    \item All fields are correctly entered.
        \begin{itemize}
            \item Input: A unique and correctly formatted e-mail address, and
all other fields are populated.
            \item Output: A user account is created and user re-directed to
the \texttt{ConMan} home page.
        \end{itemize}
\end{enumerate}

\subsubsection{Account Login}
\textbf{Motivation} A returning user is trying to log in to her \texttt{ConMan} account.

Potential cases:
\begin{enumerate}
    \item E-mail field is incorrectly entered.
        \begin{itemize}
            \item Input: An improperly formatted e-mail.
            \item Output: An error message prompts the user to check the
formatting of her e-mail address.
        \end{itemize}
    \item E-mail is not contained in the \texttt{ConMan} database.
        \begin{itemize}
            \item Input: A correctly formatted e-mail that is not associated
with a user account.
            \item Output: An error message that prompts the user that her
information could not be validated.
        \end{itemize}
    \item The password and e-mail combination do not match.
        \begin{itemize}
            \item Input: A correctly formatted e-mail that is not associated
with the given password.
            \item Output: An error message that prompts the user that her
information could not be validated.
        \end{itemize}
    \item The password or e-mail fields are left blank.
        \begin{itemize}
            \item Input: Either password or e-mail fields are left blank.
            \item Output: An error message that prompts the user to fill all
fields.
        \end{itemize}
    \item The password and e-mail combination are valid.
        \begin{itemize}
            \item Input: A correctly formatted e-mail that miraculously
matches the associated password entered by the user.
            \item Output: A user account is created and user re-directed to
the \texttt{ConMan} home page.
        \end{itemize}
\end{enumerate}

\subsubsection{Create a new team}
\textbf{Motivation} User A would like to create a new team, the
"Task-buster Trio." He has two friends who make up the trio, user B 
and user C.

Potential cases:
\begin{enumerate}
    \item Team e-mail field is incorrectly entered.
        \begin{itemize}
            \item Input: An improperly formatted e-mail.
            \item Output: An error message prompts user A to check the
formatting of his e-mail address.
        \end{itemize}
    \item The password or e-mail fields are left blank.
        \begin{itemize}
            \item Input: Either password or e-mail fields are left blank.
            \item Output: An error message that prompts the user to fill all
fields.
        \end{itemize}
    \item The e-mail field is valid and all other entries are included.
        \begin{itemize}
            \item Input: A correctly formatted e-mail and all other fields are
entered. The given password and e-mail for the Task-buster Trio does not need
to be the same as the e-mail or password combo for user A.
            \item Output: A new team is created and user A is
automatically added as the administrator.
                User A can now switch between his team and personal
contexts within the application.
        \end{itemize}
\end{enumerate}

\subsubsection{Add team members}
\textbf{Motivation} Now that Mr. Ringleader has created the "Task-buster
Trio," he would like to add his two friends, Mr. Fat-Fingers and Ms.
Forgetful.

Potential cases:
\begin{enumerate}
    \item E-mail field is incorrectly entered.
        \begin{itemize}
            \item Input: An improperly formatted e-mail.
            \item Output: An error message prompts Mr. Ringleader to check the
formatting of the e-mail address.
        \end{itemize}
    \item The e-mail field is valid but no user with that e-mail is associated
with a \texttt{ConMan} account.
        \begin{itemize}
            \item Input: A correctly formatted e-mail with no corresponding
account.
            \item Output: An error message appears indicating that the e-mail
is not found within the database.
        \end{itemize}
    \item The e-mail field is valid and all other entries are included.
        \begin{itemize}
            \item Input: A correctly formatted e-mail and all other fields are
entered.
            \item Output: The users associated with the e-mail addresses (Mr.
Fat-Fingers and Ms. Forgetful in this case) are added to the team member
list.\\
                Mr. Fat-Fingers and Ms. Forgetful can now see the "Task-buster
Trio" context within their \texttt{ConMan} accounts.
        \end{itemize}
\end{enumerate}

\subsubsection{Add tasks}
\textbf{Motivation} Ms. Forgetful cannot remember what she was supposed to do,
except that she must add a task to her \texttt{ConMan} account.
She opens the \texttt{ConMan} application and tries to add a task named
"placeholder."

Potential cases:
\begin{enumerate}
    \item Ms. Forgetful is in her basic context when she clicks "add task."
    \begin{enumerate}
        \item Ms. Forgetful leaves a required field blank.
            \begin{itemize}
                \item Input: An task-creation entry with a blank required
field.
                \item Output: An error message prompts Ms. Forgetful to
include all required fields.
            \end{itemize}
        \item The date manually entered by Ms. Forgetful was for some point in
the past.
            \begin{itemize}
                \item Input: A task creation request with an already expired
due date.
                \item Output: An error message appears telling her to pick a
future date.
            \end{itemize}
        \item The e-mail field is valid and all other entries are included.
            \begin{itemize}
                \item Input: A correctly formatted e-mail and all other fields
are entered.
                \item Output: A task that is only visible to Ms. Forgetful is
created.
            \end{itemize}
    \end{enumerate}
\item Ms. Forgetful is in her "Task-buster Trio" context when she clicks "add
task."
     \begin{enumerate}
        \item Ms. Forgetful leaves a required field blank.
            \begin{itemize}
                \item Input: An task-creation entry with a blank required
field.
                \item Output: An error message prompts Ms. Forgetful to
include all required fields.
            \end{itemize}
        \item The date manually entered by Ms. Forgetful was for some point in
the past.
            \begin{itemize}
                \item Input: A task creation request with an already expired
due date.
                \item Output: An error message appears telling her to pick a
future date.
            \end{itemize}
        \item The e-mail field is valid and all other entries are included.
            \begin{itemize}
                \item Input: A correctly formatted e-mail and all other fields
are entered.
                \item Output: A task that is visible and editable to Mr.
Taskmaster and Mr. Fat-Fingers is created.
            \end{itemize}
    \end{enumerate}
\end{enumerate}

\subsection{Test Pass / Fail Criteria}
The test will pass only if correctly formatted inputs are given.  
All other cases will display an error message to the user.  
The tests will fail if incorrect inputs are accepted by the system, or no error message is displayed.

\subsection{Test Deliverables}
A report shall be delivered stating whether the tests passed or failed. If tests failed, a SCR
(Software Change Request) shall be created to allow the development team to correct the
issues. In addition, the report should include a list of any SCRs generated for failed 
tests.

\newpage
\section{Security Testing}
\subsection{Test Risks / Issues}
No risks and issues are associated with the execution of the security tests.

\subsection{Items to be Tested / Not Tested}
Each action that a user can perform in \texttt{ConMan} must be tested.\newline

Tests:
\begin{itemize}
\item Item to test: SEC-TEST-1
\item Test description: Attempt to access any page in \texttt{ConMan} without 
being signed-in
\item Test responsibility: Angel DeCastro
\end{itemize}
\begin{itemize}
\item Item to test: SEC-TEST-2
\item Test description: Attempt to access any page in \texttt{ConMan} when 
signed-in
\item Test responsibility: Angel DeCastro
\end{itemize}
\begin{itemize}
\item Item to test: SEC-TEST-3
\item Test description: Attempt to update or delete a team that the user is 
not the admin of
\item Test responsibility: Chris Yip
\end{itemize}
\begin{itemize}
\item Item to test: SEC-TEST-3
\item Test description: Attempt to create, update or delete a task associate with 
a team that the user is not the admin of
\item Test responsibility: Chris Yip
\end{itemize}
\begin{itemize}
\item Item to test: SEC-TEST-4
\item Test description: Attempt to assign/unassign users from tasks associated 
with a team that the user is not the admin of
\item Test responsibility: Luis Retana
\end{itemize}
\begin{itemize}
\item Item to test: SEC-TEST-5
\item Test description: Attempt to create notes for tasks that the user 
is not the admin of
\item Test responsibility: Luis Retana
\end{itemize}

\subsection{Test Approach(s)}
For all tests, ensure cookies are enabled on the web browser used which must be 
either Firefox or Chrome.\newline \newline
SEC-TEST-1:
\begin{itemize}
\item Do not sign-in into \texttt{ConMan}. Attempt to navigate to every page within 
\texttt{ConMan} and record whether web application instead redirects the client 
to the \texttt{ConMan} sign-in/registration page or allows the client to navigate 
to the desired web page.
\end{itemize}
SEC-TEST-2:
\begin{itemize}
\item Sign-in into \texttt{ConMan}. Attempt to navigate to every page within 
\texttt{ConMan} and record whether web application allows the client to access the 
desired web page or not.
\end{itemize}
SEC-TEST-3, SEC-TEST-4, and SEC-TEST-5:
\begin{itemize}
\item Sign-in into \texttt{ConMan}. Navigate to the web page within \texttt{ConMan} 
that performs the action described in the test description. Record whether or not 
\texttt{ConMan} allows the action to be performed.
\end{itemize}

\subsection{Test Pass / Fail Criteria}
SEC-TEST-1:
\begin{itemize}
\item A client that is not signed-in should not be given access to any webpage 
in \texttt{ConMan}. Only the sign-in/registration page should be accessible.
\end{itemize}

SEC-TEST-2:
\begin{itemize}
\item A client that is signed-in should be given access to any webpage 
in \texttt{ConMan}.
\end{itemize}

SEC-TEST-3, SEC-TEST-4, and SEC-TEST-5:
\begin{itemize}
\item \texttt{ConMan} shoud not allow the user to perform the actions depicted in 
the test description if the user does not have the neccessary admin priviliges. 
in \texttt{ConMan}.
\end{itemize}

\subsection{Test Entry / Exit Criteria}
Tests shall begin with every major and minor release of the systems. Security testing 
will also need to be used when user-authentication methods of been modified and/or 
when the user permissions model changes. Testing should terminate when each security 
test has been completed. 

\subsection{Test Deliverables}
A report shall be delivered stating whether the tests passed or failed. If tests failed, a SCR
(Software Change Request) shall be created to allow the development team to correct the
issues. In addition, the report should include a list of any SCRs generated for failed 
tests.


\subsection{Test Suspension / Resumption Criteria}
There are no factors that would suspend security testing.

\subsection{Test Environmental / Staffing / Training Needs}
These tests require computers with Firefox and Chrome installed. These computers must
have access to ConMan on the local machine or on a test server. People responsible for
carrying out the tests must have knowledge of how to log into \texttt{ConMan} and how to 
navigate to and through every web page in \texttt{ConMan}.


\end{document}

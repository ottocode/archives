\documentclass{article}

\usepackage{fancyhdr}
\setlength{\headheight}{12pt}
\setlength{\textwidth}{17.2cm} \setlength{\textheight}{23cm}
\setlength{\topmargin}{-2.5cm} \setlength{\headsep}{1.6cm}
\setlength{\evensidemargin}{-.8cm}
\setlength{\oddsidemargin}{-.8cm}
%\pagestyle{fancy}

%set-up page dimentions
\usepackage[top=1.5 in, bottom = 1.5 in ,left = 1.5 in, right = 1.5in]{geometry}

\setlength{\parskip}{12pt}  % 12 pt = space between paragraphs
\setlength{\parindent}{0pt} % 0 pt  = indentation
\usepackage{amsmath}
\usepackage{amssymb}
\usepackage{amsthm}
\usepackage{ifthen}
\usepackage{latexsym}
\usepackage{graphicx}
\usepackage{graphics}
\usepackage{psfrag}
\usepackage{graphpap}
\usepackage{setspace}
\renewcommand{\P}{\text{P}}
\newcommand{\C}{\text{C}}


\newcommand{\natnums}{{\mathbb N}}
\newcommand{\algnums}{{\mathbb A}}
\newcommand{\rationals}{{\mathbb Q}}
\newcommand{\reals}{{\mathbb R}}
\newcommand{\norm}[1]{\left|\left|#1\right|\right|}
\newcommand{\unorm}[1]{{\left|\left|#1\right|\right|_u}}
\newcommand{\scriptR}{\mathcal{R}}
\newcommand{\scriptP}{\mathcal{P}}
\newcommand{\taggedP}{\dot{\mathcal{P}}}
\newcommand{\scriptQ}{\mathcal{Q}}
\newcommand{\taggedQ}{\dot{\mathcal{Q}}}


% Allows hyperlinks if compiled with pdflatex
\usepackage{hyperref}
\hypersetup{colorlinks}
\usepackage{color}
\definecolor{darkred}{rgb}{0.5,0,0}
\definecolor{darkgreen}{rgb}{0,0.5,0}
\definecolor{darkblue}{rgb}{0,0,0.5}
\hypersetup{ colorlinks,
                linkcolor=darkblue,
                filecolor=darkgreen,
                urlcolor=darkblue,
                citecolor=darkblue }
%hyperlink example is: \href{http://www.google.com}{google}

%add code!
\usepackage{listings}
\definecolor{mygreen}{rgb}{0,0.6,0}
\definecolor{mygray}{rgb}{0.5,0.5,0.5}
\definecolor{mymauve}{rgb}{0.58,0,0.82}
\lstset{ %
backgroundcolor=\color{white},   % choose the background color; you must add \usepackage{color} or \usepackage{xcolor}
basicstyle=\footnotesize,        % the size of the fonts that are used for the code
breakatwhitespace=false,         % sets if automatic breaks should only happen at whitespace
breaklines=true,                 % sets automatic line breaking
captionpos=b,                    % sets the caption-position to bottom
commentstyle=\color{mygreen},    % comment style
deletekeywords={...},            % if you want to delete keywords from the given language
escapeinside={\%*}{*)},          % if you want to add LaTeX within your code
extendedchars=true,              % lets you use non-ASCII characters; for 8-bits encodings only, does not work with UTF-8
frame=single,                    % adds a frame around the code
keepspaces=true,                 % keeps spaces in text, useful for keeping indentation of code (possibly needs columns=flexible)
keywordstyle=\color{blue},       % keyword style
language=C,                 % the language of the code
morekeywords={*,...},            % if you want to add more keywords to the set
numbers=left,                    % where to put the line-numbers; possible values are (none, left, right)
numbersep=5pt,                   % how far the line-numbers are from the code
numberstyle=\tiny\color{mygray}, % the style that is used for the line-numbers
rulecolor=\color{black},         % if not set, the frame-color may be changed on line-breaks within not-black text (e.g. comments (green here))
showspaces=false,                % show spaces everywhere adding particular underscores; it overrides 'showstringspaces'
showstringspaces=false,          % underline spaces within strings only
showtabs=false,                  % show tabs within strings adding particular underscores
stepnumber=2,                    % the step between two line-numbers. If it's 1, each line will be numbered
stringstyle=\color{mymauve},     % string literal style
tabsize=2,                       % sets default tabsize to 2 spaces
title=\lstname                   % show the filename of files included with \lstinputlisting; also try caption instead of title
}


\begin{document}
 
\begin{titlepage}
\Huge
\vspace{2cm}
\begin{center}
Software Requirements Specification for the \texttt{ConMan}\\
\vspace{3cm}
\Large
Authors:\\ 

\begin{tabular}{l l}\hline
Angel De Castro & Luis Retana\\
Nicholas Otto   & Christopher Yip
\end{tabular}
\vspace{1cm}

Version 0.1\\
10/18/2013
\end{center}


\end{titlepage}

\section{Introduction}
\subsection{Overview}
This change requests reflects the progress of work thus far on the \texttt{ConMan} application.
While the \texttt{ConMan} team would like to meet every software delivery target on time, quality must take priority.
For that reason, the developers propose the following modifications to the initial release of \texttt{ConMan}:
\begin{itemize}
\item The task-specific checklist feature will be delayed until future releases of \texttt{ConMan}.  
The applications will still be functional without this feature.  
Tasks contain a "notes" section which can serve a similar purpose until checklists are implemented.
\item The calendar view displaying all upcoming tasks will be delayed until future releases of \texttt{ConMan}.
The applications will still be functional without this feature.  
Each team has a "view-tasks" section that allows the team members to view all of the tasks assigned to that team.
While not as friendly as a calendar view, this will serve the same purpose.
\end{itemize}

\subsection{Related Documents}
This change request will reference portions of the Software Requirements Specification and the Software Design Document for the \texttt{ConMan} application.

\newpage
\section{Proposed Changes}
\subsection{Checklists}
\subsubsection{Impact}
It is proposed that the checklist feature be delayed until future releases of \texttt{ConMan}.  
As an alternative feature until checklists are implemented, the tasks feature of the application contains a "notes" section that will allow team members who are assigned a task to dynamically update the tasks with their notes, progress reports, and other concerns.
Managers of teams are encouraged to devise a standard for adding to a task's notes section in order to effectively utilize that feature in a way meaningful to their organization.
By delaying the checklist feature, the marketing team may find opportunities to build anticipation and excitement about future developments in the application.

\subsubsection{Documents Affected}
Checklists were initially described in the Software Requirements Specification sections 3.1 and 4.3.  Additionally checklists were referred to in the Software Design Document sections 1.2, 2.3, 2.4, 3.1, 4.2.2, and 4.2.3.

\subsubsection{Change Implementation}
The design of \texttt{ConMan} is fairly modular and the checklist feature can be left unimplemented without affecting the working components of the rest of \texttt{ConMan}.
The task view as shown in the Software Design Document section 4.2.2 will no longer have the option to add/view or delete checklists.
The database tables related specifically to checklists will also not be created.
No other conflicts are foreseen at this time.

\newpage
\subsection{Calendar View}
\subsubsection{Impact}
It is proposed that the calendar view feature be delayed until future releases of \texttt{ConMan}.
The original intention for the home-screen view for a team was to display a calendar with all relevant tasks and deadlines.
Unfortunately, this feature will not be ready in time for the initial release and the development team proposes the integration of the calendar view be delayed to ensure over-all quality of \texttt{ConMan}.
As an alternative, each team as an option to "view tasks" will will display all of the tasks associated with that team in a list view as opposed to a calendar view.
By delaying the calendar feature, the marketing team may find opportunities to build anticipation and excitement about future developments in the application.


\subsubsection{Documents Affected}
The calendar was initially described in the Software Requirements Specification sections 2.3, 3.1, and 4.1.
Additional references to the calendar are found in the Software Design Document section 4.2.1.

\subsubsection{Change Implementation}
The calendar feature of \texttt{ConMan} is to be implemented entirely as a user interface feature.
By delaying the release of the calendar view, no changes to the database design or data access functions are required.
All links to the calendar view will need to be removed from other pages within the application.
No other conflicts are foreseen at this time.

\end{document}
